Se ha demostrado que el sistema implementado en Simulink para realizar la adquisición y decodificación funciona adecuadamente, y gracias a la métrica de correlación obtenida se puede fundamentar su buen funcionamiento.

El protocolo de registro, a pesar de no ser selectivo en su totalidad, permite distinguir en la mayoría de los casos entre actividad sEMG relacionada a un movimiento u otro. Sin embargo, se ha observado que se deben de dar indicaciones claras al sujeto experimental, ya que se a observado que al realizar contracciones fuertes en cualquier movimiento, la actividad de sEMG en ambos canales es muy similar.

El esquema de filtrado diseñado para los registros de sEMG es útil ya que permite obtener una señal de sEMG en la que se pueden distinguir las contracciones de los momentos de reposo. En cuanto a los descriptores de amplitud se optará por utilizar el valor RMS debido a que proporciona valores más grandes que el valor ARV/MAV. Cabe destacar que se ha observado que, en cuanto a morfología de las señales obtenidas tras aplicar el descriptor, se obtiene una forma muy similar para ambos descriptores. Un punto importante relacionado al procesamiento es el método de suavizado; este genera una onda suave que permite obtener un mapeo con mejor estabilidad que sin suavizar, sin embargo, este procedimiento de suavizado utiliza un buffer de 10 muestras de valor RMS, recordemos que cada muestra de RMS se obtiene a partir del procesamiento de ventanas de 0.1 segundos de sEMG, lo cuál implica que se está generando un retraso total de 1 segundo en la actualización del parámetro a mapear, lo cual nos coloca en el límite del tiempo prometido; para solucionar esto se probaran ventanas más cortas.

El mapeo lineal es útil hasta cierto punto, pero si se opta por utilizar dicho mapeo es muy probable que sea necesario utilizar una lógica de control adicional que ayude a obtener un mapeo más estable. Adicional a esto aún está como alternativa a probar el obtener una relación entre los descriptores y la amplitud de la corriente de estimulación a través de una calibración a 5 posturas de la mano (cierre total de la mano, cierre parcial, descanso, apertura parcial de mano y apertura total).

Hay que destacar que todo el trabajo realizado se ha probado de forma exitosa con el dispositivo OpenBCI debido a que el prototipo presenta problemas en el circuito encargado de la conversión analógico-digital. Afortunadamente todo lo desarrollado tiene la posibilidad de migrarse al dispositivo de forma sencilla, por lo cual en cuanto se solucionen los problemas con el prototipo se migrará todo a este.