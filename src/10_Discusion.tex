%%DISCUSIÓN
El trabajo realizado en este proyecto se espera sea tomado en cuenta para generar mejores técnicas para rehabilitación de pacientes con discapacidad de miembro superior, basando dichas técnicas en una neuroprótesis para rehabilitación basada en FES. Este proyecto plantea las bases para realizar una adecuada implementación de una aplicación de estimulación eléctrica en lazo cerrado, a partir de señales sEMG del miembro no afectado en pacientes con hemiplejia por ACV (Accidente Cerebro-Vascular). Los principales resultados obtenidos se discuten a continuación, junto a las limitaciones del trabajo y posibles temas para trabajar a futuro.

\subsubsection*{Adquisición y decodificación de datos}
El subsistema diseñado para realizar la adquisición y decodificación de datos tiene la particularidad de que puede ser utilizado para cualquier dispositivo de adquisición que utilice un chip ADS1299, como el prototipo de adquisición en desarrollo en la División de Investigación Médica del INR-LGII, sólo son necesarios pequeños cambios en la selección de los bytes correspondientes a cada canal. Un problema que tiene este subsistema se encuentra en el bloque responsable de realizar la solicitud de muestras al dispositivo de adquisición, ya que es un bloque perteneciente al \emph{Instrument Control Toolbox} de Simulink\textregistered, por lo cual si no se cuenta con dicho toolbox el sistema no será funcional. Una posible mejora a este subsistema sería el diseño de un bloque responsable de la solicitud de muestras implementado en algún lenguaje de bajo nivel, esto podría hacer al sistema flexible y veloz, ya que actualmente el bloque de solicitud realiza una comunicación con MATLAB\textregistered para poder establecer una conexión serial con el dispositivo de adquisición, proceso que puede estar generando algún retraso dentro de todo el sistema. Un aspecto importante que podría ayudar a rastrear el origen del retardo existente actualmente en el sistema desarrollado en este proyecto, es la medición del retardo que genera dicho subsistema por sí solo, esto podría ayudar a determinar los puntos de trabajo para una mejora de este sistema desarrollado.

\subsubsection*{Protocolo para registro de sEMG}
El protocolo descrito en este proyecto se presta a errores humanos al momento de ubicar el lugar adecuado para la colocación de electrodos, por lo cual no se garantiza una repetibilidad del 100$\%$ en los registros. Se propone realizar un estudio donde se analice la actividad mioeléctrica en distintas posiciones del brazo en diversos sujetos, buscando obtener una estandarización en el posicionamiento de electrodos para aplicaciones similares a la desarrollada en este proyecto.

\subsubsection*{Procesamiento de sEMG}
Actualmente todo el procesamiento de las señales de sEMG se lleva a cabo por ventanas no traslapadas de adquisición, este proceso genera un retardo natural definido por la longitud de la ventana analizada, por lo cual el realizar un procesamiento con ventanas traslapadas o bien muestra a muestra podría disminuir este retardo natural. La implementación de un filtro de mediana móvil resultó de gran utilidad para conseguir una envolvente suave que sirviera como señal de control, sin embargo existen métodos como la regla trapezoidal que podrían arrojar resultados similares, por lo cual se podrían implementar algún otro método y determinar si dicho método disminuye el retardo del sistema. Un aspecto sumamente importante en el procesamiento es el hecho de que no se trabajó con señales de sEMG normalizadas, esto podría estar afectando al desempeño del sistema y sería una buena idea implementar una aplicación similar evaluando el desempeño utilizando sEMG normalizado y no normalizado. Esto último es particularmente importante, ya que la actividad mioeléctrica que presenta un sujeto en diversas sesiones puede cambiar de manera significativa, y claramente la actividad mioeléctrica tampoco será similiar entre diferentes sujetos, por lo que implementar el sistema utilizando señales de sEMG normalizado podría ayudar a estandarizar el sistema de control diseñado.

\subsubsection*{Sistema de control}
Recordando que el sistema de control consta de dos grandes partes: 1)Máquina de estados finitos para la identificación de movimientos; 2)Control proporcional para realizar modulación sEMG-FES. Se puede comentar lo siguiente:

%Actualmente el esquema de control se divide en dos grandes partes: 1)Máquina de estados finitos para la identificación de movimientos; 2)Ecuación lineal para realizar el mapeo sEMG-FES.

En cuanto a la identificación de movimientos, se considera que la implementación de un clasificador basado en una FSM que cambia de estado según se superen determinados valores de umbrales no es la mejor forma para realizar una clasificación, pero quizás sí una de las más fáciles. Este clasificador demostró ciertos problemas en identificar cambios de estado visualmente notorios, por ejemplo, en los movimientos incompletos de apertura o cierre existían momentos en los cuales el clasificador no identificaba el movimiento de forma adecuada a pesar de que visualmente se notara un cambio en la envolvente. Adicional a esto, existe también la posibilidad de que los propios umbrales puedan estar generando un retardo en el tiempo de inicio de la estimulación eléctrica, ya que habrá movimientos incompletos que no logren pasar ese umbral, y por lo tanto no activar la estimulación eléctrica; una buena idea sería probar disminuir los umbrales para lograr la activación de la estimulación con movimientos ligeros. Otro aspecto importante de este clasificador implementado, es la imposibilidad de compensar la fatiga muscular. Implementar un clasificador robusto basado en algún algoritmo de inteligencia artificial o LDA podría ser de mayor utilidad para una aplicación que se fuera utilizar para rehabilitación.

En cuanto al control proporcional encargado de la modulación de la amplitud FES, hay que destacar que este se obtiene a partir de dos umbrales obtenidos en la etapa de calibración, y considerando que el sEMG no presenta un comportamiento lineal, es probable que este método para la obtención de la ecuación del control proporcional no pueda realizar un seguimiento preciso a los cambios existentes en las señales de sEMG. Una mejora a esta parte podría ser la realización de una calibración con más de dos puntos, y a partir de ellos obtener una regresión lineal, o en su momento introducir algún bloque de control que logre representar la relación existente entre las señales sEMG y la fuerza ejercida o el torque muscular. Cabe destacar que debido a que la técnica de retroalimentación utilizada para este proyecto es el biofeedback, no existe como una señal de retroalimentación que se esté considerando dentro del control proporcional, y recordemos que la definición de este control estable que la señal control debe ser una señal de error obtenida a partir de la diferencia entre la señal ideal y la señal real, por ejemplo, la señal de sEMG del miembro no afectado y la señal de sEMG del miembro afectado. El agregar una señal de retroalimentación que pueda ser considerada dentro del sistema de control probablemente mejorará el desempeño de este.

Relacionado a la prueba de la tarea funcional, el hecho de que se lograra llevar a cabo con éxito es un indicador del potencial que tienen aplicaciones basadas en este proyecto para terapias de rehabilitación, dando a los pacientes la posibilidad de realizar tareas comunes de su vida diaria. Se espera que con las correctas adecuaciones, el sistema diseñado en este proyecto pueda ser de utilidad en sujetos con hemiplejia. Adicionalmente, sería bueno realizar una variante de esta prueba donde se llevara a cabo el mismo procedimiento pero mediante un experimento cruzado (un sujeto controla mientras otro sujeto es estimulado), y junto con algún algoritmo de visión por computadora poder determinar el retardo existente en esta prueba, para así tener una métrica más sobre el desempeño del sistema.

%Cabe destacar que al final del desarrollo de este proyecto se tuvo la posibilidad de probar la utilidad del sistema dentro de una situación de la vida real. Esta prueba se realizó a un sujeto sano, al cual se le otorgó la tarea de tomar un objeto cilíndrico con su mano derecha pero utilizando solamente la corriente eléctrica modulada por su brazo izquierdo. El sujeto pudo tomar el objeto de forma eficaz, y realizando movimientos del hombro logró levantar el objeto y trasladarlo a un lugar diferente a donde tomó el objeto. Con esto se logró demostrar que aplicaciones basadas en este proyecto pueden ser de utilidad en terapias de rehabilitación, dando a los pacientes la posibilidad de realizar tareas comunes de su vida diaria. Se espera que con las correctas adecuaciones, el sistema diseñado en este proyecto pueda ser de utilidad en sujetos con hemiplejia.

%Se ha demostrado que el sistema implementado en Simulink para realizar la adquisición y decodificación funciona adecuadamente, y gracias a la métrica de correlación obtenida se puede fundamentar su buen funcionamiento.
%
%El protocolo de registro, a pesar de no ser selectivo en su totalidad, permite distinguir en la mayoría de los casos entre actividad sEMG relacionada a un movimiento u otro. Sin embargo, se ha observado que se deben de dar indicaciones claras al sujeto experimental, ya que se a observado que al realizar contracciones fuertes en cualquier movimiento, la actividad de sEMG en ambos canales es muy similar.
%
%El esquema de filtrado diseñado para los registros de sEMG es útil ya que permite obtener una señal de sEMG en la que se pueden distinguir las contracciones de los momentos de reposo. En cuanto a los descriptores de amplitud se optará por utilizar el valor RMS debido a que proporciona valores más grandes que el valor ARV/MAV. Cabe destacar que se ha observado que, en cuanto a morfología de las señales obtenidas tras aplicar el descriptor, se obtiene una forma muy similar para ambos descriptores. Un punto importante relacionado al procesamiento es el método de suavizado; este genera una onda suave que permite obtener un mapeo con mejor estabilidad que sin suavizar, sin embargo, este procedimiento de suavizado utiliza un buffer de 10 muestras de valor RMS, recordemos que cada muestra de RMS se obtiene a partir del procesamiento de ventanas de 0.1 segundos de sEMG, lo cuál implica que se está generando un retraso total de 1 segundo en la actualización del parámetro a mapear, lo cual nos coloca en el límite del tiempo prometido; para solucionar esto se probaran ventanas más cortas.
%
%El mapeo lineal es útil hasta cierto punto, pero si se opta por utilizar dicho mapeo es muy probable que sea necesario utilizar una lógica de control adicional que ayude a obtener un mapeo más estable. Adicional a esto aún está como alternativa a probar el obtener una relación entre los descriptores y la amplitud de la corriente de estimulación a través de una calibración a 5 posturas de la mano (cierre total de la mano, cierre parcial, descanso, apertura parcial de mano y apertura total).
%
%Hay que destacar que todo el trabajo realizado se ha probado de forma exitosa con el dispositivo OpenBCI debido a que el prototipo presenta problemas en el circuito encargado de la conversión analógico-digital. Afortunadamente todo lo desarrollado tiene la posibilidad de migrarse al dispositivo de forma sencilla, por lo cual en cuanto se solucionen los problemas con el prototipo se migrará todo a este.