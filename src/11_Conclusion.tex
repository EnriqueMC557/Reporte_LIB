Aquí va una conclusión interesante :)

La aplicación sEMG-FES diseñada en este trabajo, demostró que pese a sus 

En conclusión, la aplicación sEMG-FES diseñada en este trabajo logró cumplir el objetivo principal del mismo, y además, pese a las limitaciones del sistema, se logró demostrar que el enfoque utilizado en este trabajo puede ser de utilidad en aplicaciones diseñadas para terapia de rehabilitación.

Relacionado al protocolo de registro se concluye que este es susceptible a errores humanos, por lo cuál es recomendable utilizar otro protocolo o seguir guías clínicas como las ya mencionadas guías del SENIAM.

En cuanto al algoritmo para clasificación de movimientos, se destaca que el algoritmo diseñado basado en umbrales no fue la mejor solución. Un algoritmo basado en inteligencia artificial probablemente arroje mejores resultados.

Por último, relacionado al mapeo sEMG-FES, se obtuvo una solución rápida y sencilla, la cuál se puede adaptar a los tipos de señales con las que se cuente, sin embargo, al no ser lineal el sEMG es probable que este este método no sea el adecuado para dicha señal.
 