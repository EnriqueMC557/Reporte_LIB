\section{General}
Desarrollar los bloques de software y algoritmos para implementar el modo de operación en lazo cerrado de una neuroprótesis de miembro superior basada en estimulación eléctrica funcional, integrando un sistema prototipo de adquisición de biopotenciales y un estimulador comercial, con un enfoque de control que permita la operación en línea.

\section{Específicos}

\begin{itemize}
	\item Evaluar el protocolo de comunicación diseñado previamente para la transmisión de datos entre el prototipo de adquisición y la computadora.
	\item Desarrollar un algoritmo de que permita la modulación de los parámetros de estimulación eléctrica para restaurar la función de la pinza gruesa, a partir de la señal de electromiografía de superficie (sEMG) correspondiente al antebrazo contralateral y un sensor de fuerza como retroalimentación.
	\item Implementar un esquema de control reportado en la literatura que sea útil para la aplicación.
\end{itemize}