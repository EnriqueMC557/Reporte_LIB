%INTRODUCCIÓN

En la división de Investigación en Ingeniería Médica del Instituto Nacional de Rehabilitación ``Luis Guillermo Ibarra Ibarra'' (INR-LGII) se llevan a cabo diversos proyectos de investigación y desarrollo tecnológico, los cuales buscan desarrollar tecnología que permita aplicar nuevas técnicas de terapia de rehabilitación para personas con discapacidad motriz, o bien mejorar las técnicas ya existentes. Tal es el caso del proyecto de desarrollo de una neuroprótesis basada en estimulación eléctrica funcional (FES, por sus siglas en inglés) para rehabilitación de miembro superior, que busca satisfacer las necesidades del INR-LGII en cuestiones de terapia e investigación.

Para dicho proyecto el INR-LGII desarrolló una interfaz gráfica de usuario (GUI, por sus siglas en inglés) que permite controlar los parámetros del sistema de estimulación eléctrica RehaStim 2 (HASOMED GmbH., Magdeburgo, Alemania) y el sistema de adquisición de biopotenciales Cyton Board (OpenBCI Inc., Nueva York, E.U.A.), siendo estos tres (GUI, RehaStim 2 y Cyton Board) los principales elementos de la neuroprótesis del INR-LGII. Una primera aplicación de la neuroprótesis se encuentra funcionando y se ha utilizado con sujetos sanos y pacientes del INR-LGII, aplicándoles estimulación eléctrica para generar movimientos de miembro superior. Sin embargo, actualmente el sistema opera en lazo abierto, es decir, la secuencia de estimulación se diseña y se aplica sin considerar información relevante del movimiento generado y variables relacionadas. En el estado actual del sistema, los parámetros de la estimulación eléctrica son controlados por el experimentador, quien los adapta hasta obtener el patrón específico útil para el sujeto en rehabilitación, lo cual causa una dependencia del experimentador para la operación del sistema, y limita su utilidad para objetivos de rehabilitación, ya que no se consideran variables intrínsecas del paciente, como lo son: la intención de movimiento o actividad residual, que podían contribuir a la modulación de los parámetros de estimulación y el movimiento resultante.

%Adicional a la GUI desarrollada en el INR, incluyendo la implementación de un funcionamiento en lazo abierto, se desarrolló un prototipo de adquisición de biopotenciales que permitiera tener un control total sobre los parámetros de adquisición (frecuencia de muestreo, ganancia y filtros) y que además fuera compatible con la aplicación en lazo abierto antes mencionada.

%\section{Justificación}
%{\color{blue}JUSTIFICACIÓN\\}

Debido a que la neuroprótesis desarrollada en el INR-LGII se ha utilizado en aplicaciones de lazo abierto donde los sistemas involucrados trabajan sin tener retroalimentación entre ellos, surge la necesidad de implementar alguna aplicación que permita la interacción entre sistemas y que además permita la participación del paciente de forma cuantitativa en la terapia.

Por ello, este proyecto plantea desarrollar una aplicación en lazo cerrado que permita la participación activa del paciente (a partir de señales de sEMG del brazo sano) en la terapia de rehabilitación basada en FES, llevando a cabo una modulación de la estimulación eléctrica que permita la ejecución, en el brazo afectado, de los movimientos involucrados en el agarre de un objeto (flexión y extensión de los dedos). Un sistema con estas características permitiría al sujeto tener el control sobre los movimientos de la mano contralateral, y además disminuiría la dependencia del experimentador en el proceso de la terapia.

%\section{Planteamiento del problema}
{\color{red}PROBLEMA\\}

Para realizar el diseño de la aplicación se utilizarán dos canales de electromiografía de superficie (sEMG, por sus siglas en inglés) para la extracción de algún rasgo descriptivo de su amplitud, el cuál servirá para realizar el diseño de un esquema de control que permita identificar el tipo de movimiento a realizar (flexión o extensión de los dedos) y a su vez sirva como modulador de la amplitud de corriente eléctrica a inyectar en el miembro contrario, buscando lograr un funcionamiento en tiempo real.

%\section{Hipótesis}
%{\color{blue}HIPÓTESIS\\}

Por lo tanto, al integrar los diferentes bloques de un sistema de estimulación eléctrica funcional dentro de una plataforma de software con herramientas de tiempo real, se logrará implementar una aplicación FES en lazo cerrado que permita llevar a cabo en línea la adquisición y procesamiento de señales de sEMG, como parte de un esquema de control contralateral de movimientos de la mano.

%Por lo tanto, la hipótesis de este proyecto consiste en que al integrar los diferentes bloques de un sistema de estimulación eléctrica funcional dentro de una plataforma de software con herramientas de tiempo real, se logrará implementar un lazo cerrado que permita llevar a cabo las tareas de adquisición de sEMG, procesamiento de este y aplicación de estimulación eléctrica en línea.

%La hipótesis de este proyecto consiste en que al integrar los diferentes bloques de un sistema de estimulación eléctrica funcional

%\section{Objetivos}
%{\color{blue}OBJETIVOS\\}

Por ello, el objetivo general de este proyecto consiste en diseñar e implementar un esquema de control contralateral para miembro superior, que permita la modulación de la amplitud de estimulación eléctrica a partir de señales de sEMG.

Teniendo como objetivos particulares los siguientes:

\begin{itemize}
	\item Desarrollar y evaluar un bloque de adquisición dentro de Simulink que permita la recepción de datos seriales de un dispositivo de multicanal de adquisición de biopotenciales.
	\item Desarrollar un algoritmo que permita la identificación de los movimientos de flexión y extensión de la mano a través de señales de sEMG.
	\item Diseñar e implementar un esquema de control que permita la modulación continua de la amplitud de la estimulación eléctrica para control del movimiento de la mano, en combinación con el algoritmo identificador de movimientos.
\end{itemize}
