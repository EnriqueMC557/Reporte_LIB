%INTRODUCCIÓN

En la división de Investigación en Ingeniería Médica del Instituto Nacional de Rehabilitación ``Luis Guillermo Ibarra Ibarra'' (INR) se llevan a cabo diversos proyectos, entre los cuales se desarrolla tecnología que permita aplicar nuevas técnicas de terapia de rehabilitación para personas con discapacidad motriz, o bien mejorar las técnicas ya existentes. Tal es el caso del proyecto de desarrollo de una neuroprótesis basada en estimulación eléctrica funcional (FES, por sus siglas en inglés) para rehabilitación de miembro superior, que busca satisfacer las necesidades del INR en cuestiones de terapia e investigación.

Para dicho proyecto el INR desarrolló una interfaz gráfica de usuario (GUI, por sus siglas en inglés) que permite controlar los parámetros de los dispositivos comerciales: RehaStim 2 para la estimulación eléctrica y Cyton Board para adquisición de biopotenciales, que la neuroprótesis necesita para su correcto funcionamiento. Una primera implementación de la neuroprótesis se encuentra funcionando y se han realizado las primeras pruebas con sujetos sanos y pacientes del INR, aplicándoles estimulación eléctrica para generar movimientos de miembro superior. Sin embargo, actualmente el sistema opera en lazo abierto, es decir, se determina la secuencia de estimulación a partir de la información de entrada, sin medir la información de la salida, en este caso, el movimiento generado y sus variables relacionadas. En el estado actual del sistema, los parámetros de la estimulación eléctrica son controlados por el experimentador, quien los adapta hasta obtener el patrón específico útil para el sujeto en rehabilitación. Esto causa una dependencia del experimentador para la operación del sistema, y limita su utilidad para objetivos de rehabilitación, al no considerar variables intrínsecas del paciente, como lo son: la intención de movimiento o actividad residual, que podían contribuir a la modulación de los parámetros de estimulación y el movimiento resultante.

Adicional a la IGU desarrollada en el INR, incluyendo la implementación de un funcionamiento en lazo abierto, se desarrolló un prototipo de adquisición de biopotenciales que permitiera tener un control total sobre los parámetros de adquisición (frecuencia de muestreo, ganancia y filtros) y que además fuera compatible con la aplicación en lazo abierto antes mencionada.

{\color{blue}JUSTIFICACIÓN\\}

Debido a que los sistemas de software y hardware desarrollados en el INR se han utilizado para trabajar en aplicaciones donde dichos sistemas trabajan sin tener interacción entre ellos, surge la necesidad de implementar alguna aplicación que permita la interacción entre sistemas y que además permita la participación del paciente de forma cuantitativa en la terapia.

Por ello, este proyecto plantea desarrollar una aplicación en lazo cerrado que permita lograr una intervención basada en la actividad, es decir, que involucre la actividad voluntaria del paciente para lograr la modulación de la estimulación eléctrica y esta a su vez permita la repetición de movimientos en un contexto relevante para la rehabilitación del paciente, dotándolo así de control sobre los movimientos en su rehabilitación. Además, se logrará disminuir la dependencia del experimentador al momento de realizar la terapia de estimulación.

{\color{blue}PROBLEMA\\}

Para lograr la aplicación en lazo cerrado se utilizará el prototipo de adquisición de biopotenciales desarrollado en el INR  para la adquisición de dos canales de electromiografía de superficie (sEMG, por sus siglas en inglés), de los cuales se extraerá algún rasgo descriptivo de su amplitud, el cual servirá para modular la intensidad de estimulación eléctrica que será suministrada por el sistema RehaStim 2. Con esto, se tendrá que diseñar e implementar bloques de adquisición y procesamiento dentro de Simunlink que permitan un funcionamiento en tiempo real.

{\color{blue}HIPÓTESIS\\}

Por lo tanto, la hipótesis de este proyecto consiste en que al integrar los diferentes bloques de un sistema de estimulación eléctrica funcional dentro de una plataforma de software con herramientas de tiempo real, se logrará implementar un lazo cerrado que permita la adquisición de sEMG, procesamiento y estimulación eléctrica, con un tiempo de latencia en la actualización de los parámetros de estimulación menor a 1 segundo.

{\color{blue}OBJETIVOS\\}

Para conseguir esto, se plantearon como objetivos del proyecto los siguientes:

\begin{itemize}
	\item Desarrollar un bloque de adquisición dentro de Simulink que permita la recepción de datos desde la computadora.
	\item Evaluar la calidad de la transmisión de datos entre el prototipo de adquisición y la computadora.
	\item Desarrollar un algoritmo que permita la modulación de los parámetros de estimulación eléctrica, para restaurar la función de la pinza gruesa, a partir de señales de sEMG correspondientes al antebrazo contralateral.
	\item Implementar un esquema de control reportado en la literatura que sea útil para la aplicación.
\end{itemize}
