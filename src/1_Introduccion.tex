%INTRODUCCIÓN

En la división de Investigación en Ingeniería Médica del Instituto Nacional de Rehabilitación ``Luis Guillermo Ibarra Ibarra'' (INR) se llevan a cabo diversos proyectos de investigación y desarrollo tecnológico, los cuales buscan desarrollar tecnología que permita aplicar nuevas técnicas de terapia de rehabilitación para personas con discapacidad motriz, o bien mejorar las técnicas ya existentes. Tal es el caso del proyecto de desarrollo de una neuroprótesis basada en estimulación eléctrica funcional (FES, por sus siglas en inglés) para rehabilitación de miembro superior, que busca satisfacer las necesidades del INR en cuestiones de terapia e investigación.

Para dicho proyecto el INR desarrolló una interfaz gráfica de usuario (GUI, por sus siglas en inglés) que permite controlar los parámetros de los dispositivos comerciales RehaMove 2 (Hasomed GmbH, Alemania) para estimulación eléctrica, y Cyton Board (OpeBCI Inc, E.E.U.U.) para adquisición de biopotenciales, que la neuroprótesis necesita para su correcto funcionamiento. Una primera aplicación de la neuroprótesis se encuentra funcionando y se utilizado con sujetos sanos y pacientes del INR, aplicándoles estimulación eléctrica para generar movimientos de miembro superior. Sin embargo, actualmente el sistema opera en lazo abierto, es decir, se determina la secuencia de estimulación sin considerar información relevante como el movimiento generado y sus variables relacionadas. En el estado actual del sistema, los parámetros de la estimulación eléctrica son controlados por el experimentador, quien los adapta hasta obtener el patrón específico útil para el sujeto en rehabilitación, esto causa una dependencia del experimentador para la operación del sistema, y limita su utilidad para objetivos de rehabilitación, ya que no se consideran variables intrínsecas del paciente, como lo son: la intención de movimiento o actividad residual, que podían contribuir a la modulación de los parámetros de estimulación y el movimiento resultante.

%Adicional a la GUI desarrollada en el INR, incluyendo la implementación de un funcionamiento en lazo abierto, se desarrolló un prototipo de adquisición de biopotenciales que permitiera tener un control total sobre los parámetros de adquisición (frecuencia de muestreo, ganancia y filtros) y que además fuera compatible con la aplicación en lazo abierto antes mencionada.

\section{Justificación}
%{\color{blue}JUSTIFICACIÓN\\}

Debido a que la neuroprótesis desarrollada en el INR se ha utilizado para trabajar en aplicaciones donde los sistemas involucrados para su funcionamiento trabajan sin tener interacción entre ellos, surge la necesidad de implementar alguna aplicación que permita la interacción entre sistemas y que además permita la participación del paciente de forma cuantitativa en la terapia.

Por ello, este proyecto plantea desarrollar una aplicación en lazo cerrado que involucre la actividad voluntaria del paciente para lograr la modulación de la estimulación eléctrica y esta a su vez permita la repetición de los movimientos relacionados al agarre de un objeto (flexión y extensión de los dedos), dotándolo así de control sobre los movimientos en su rehabilitación y disminuyendo la dependencia del experimentador al momento de realizar la terapia de estimulación.

\section{Planteamiento del problema}
%{\color{blue}PROBLEMA\\}

Para realizar el diseño de la aplicación se utilizarán dos canales de electromiografía de superficie (sEMG, por sus siglas en inglés) para la extracción de algún rasgo descriptivo de su amplitud, el cuál servirá para realizar el diseño de un esquema de control que permita identificar el tipo de movimiento a realizar (flexión o extensión de los dedos) y a su vez sirva como modulador de la amplitud de corriente eléctrica a inyectar en el miembro contrario, buscando lograr un funcionamiento en tiempo real.

\section{Hipótesis}
%{\color{blue}HIPÓTESIS\\}

Por lo tanto, la hipótesis de este proyecto consiste en que al integrar los diferentes bloques de un sistema de estimulación eléctrica funcional dentro de una plataforma de software con herramientas de tiempo real, se logrará implementar un lazo cerrado que permita llevar a cabo las tareas de adquisición de sEMG, procesamiento de este y aplicación de estimulación eléctrica, con un tiempo de latencia en la actualización de los parámetros de estimulación menor a 1 segundo.

\section{Objetivos}
%{\color{blue}OBJETIVOS\\}

Por ello, el objetivo de este proyecto se centra en diseñar e implementar un esquema de control que permita la modulación de la amplitud de estimulación eléctrica en tiempo real.

Teniendo como objetivos particulares los siguientes:

\begin{itemize}
	\item Desarrollar un bloque de adquisición dentro de Simulink que permita la recepción de datos desde la computadora.
	\item Evaluar la calidad del funcionamiento del bloque de adquisición diseñado.
	\item Desarrollar un algoritmo que permita la identificación de los movimientos de flexión y extensión de dedos.
	\item Diseñar e implementar un esquema de control que permita la modulación de la amplitud de la estimulación eléctrica a través de señales de sEMG.
\end{itemize}
