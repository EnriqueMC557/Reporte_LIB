%DISCUSIÓN
Este proyecto plantea las bases para implementar una aplicación de estimulación eléctrica en lazo cerrado de miembro superior, a partir de señales sEMG del miembro no afectado en pacientes con hemiplejia por ACV (Accidente Cerebro-Vascular). Los principales resultados obtenidos se discuten en las siguientes secciones, junto a las limitaciones del trabajo y posibles temas para trabajar a futuro. Se espera que el trabajo realizado en este proyecto sea tomado en cuenta para generar mejores técnicas para rehabilitación de pacientes con discapacidad de miembro superior, aplicando dichas técnicas en una neuroprótesis para rehabilitación basada en FES.

\subsubsection*{Adquisición y decodificación de datos}
El subsistema diseñado para realizar la adquisición y decodificación de datos tiene la particularidad de que puede ser utilizado para cualquier dispositivo de adquisición que utilice un chip ADS1299. Tal es el caso del prototipo de adquisición en desarrollo en la División de Investigación Médica del INR-LGII, donde sólo son necesarios pequeños cambios en la selección de los bytes correspondientes a cada canal. Una limitación de este subsistema se encuentra en el bloque responsable de realizar la solicitud de muestras al dispositivo de adquisición, ya que es un bloque perteneciente al \emph{Instrument Control Toolbox} de Simulink\textregistered, por lo cual si no se cuenta con dicho toolbox el sistema no será funcional.

Una posible mejora a este subsistema sería el diseño de un bloque responsable de la solicitud de muestras implementado en algún lenguaje de bajo nivel, esto podría hacer al sistema flexible y veloz, ya que actualmente el bloque de solicitud realiza una comunicación con MATLAB\textregistered \; para poder establecer una conexión serial con el dispositivo de adquisición, proceso que puede estar generando algún retraso dentro de todo el sistema. Un aspecto importante que podría ayudar a rastrear el origen del retardo existente actualmente en el sistema desarrollado en este proyecto, es la medición del retardo que genera dicho subsistema por sí solo, esto podría ayudar a determinar los puntos de trabajo para una versión mejorada.

\subsubsection*{Protocolo para registro de sEMG}
El desempeño del protocolo descrito en este proyecto puede ser afectado por errores humanos al momento de ubicar el lugar adecuado para la colocación de electrodos, por lo cual no se garantiza al 100\% una repetibilidad en los registros. Se propone realizar un estudio donde se analice la actividad mioeléctrica en distintas posiciones del brazo en diversos sujetos, buscando obtener una estandarización en el posicionamiento de electrodos para aplicaciones similares a la desarrollada en este proyecto. Tomar en cuenta para dicho estudio las guías del SENIAM (Surface ElectroMyoGraphy for the Non-Invasive Assessment of Muscles) también puede ayudar a mejorar la repetibilidad en los registros.

\subsubsection*{Procesamiento de sEMG}
Actualmente todo el procesamiento de las señales de sEMG se lleva a cabo por ventanas no traslapadas de adquisición. Este proceso genera un retardo natural definido por la longitud de la ventana analizada, por lo cual el realizar un procesamiento con ventanas traslapadas o bien muestra a muestra podría disminuir este retardo natural. Por otra parte, la implementación de un filtro de mediana móvil resultó de gran utilidad para conseguir una envolvente suave que sirviera como señal de control, sin embargo existen métodos como la regla trapezoidal que podrían arrojar resultados similares, por lo cual se podrían implementar algún otro método y determinar si disminuye el retardo del sistema.

Un aspecto sumamente importante en el procesamiento es el hecho de que no utilizó método alguno de normalización de las señales sEMG. Esto podría estar afectando al desempeño del sistema y sería una buena idea implementar una aplicación similar evaluando el desempeño utilizando sEMG normalizado y no normalizado. Esto último es particularmente importante, ya que la actividad mioeléctrica que presenta un sujeto en diversas sesiones puede cambiar de manera significativa, y claramente la actividad mioeléctrica tampoco será similiar entre diferentes sujetos, por lo que implementar el sistema utilizando señales de sEMG normalizado podría ayudar a estandarizar el sistema de control diseñado.

\subsubsection*{Sistema de control}
Recordando que el sistema de control sEMG-FES consta de dos grandes partes: 1)Máquina de estados finitos para la identificación de movimientos; 2)Control proporcional para modulación sEMG-FES. Se puede comentar lo siguiente:

En cuanto a la identificación de movimientos, se considera que la implementación de un clasificador basado en una FSM que cambia de estado según se superen determinados valores de umbrales no es la mejor forma para realizar una clasificación, pero quizás sí una de las más fáciles. Este clasificador demostró ciertos problemas en identificar cambios de estado que eran notables incluso a simple vista, por ejemplo, en los movimientos incompletos de apertura o cierre existían momentos en los cuales el clasificador no identificaba el movimiento de forma adecuada a pesar de que visualmente se notara un cambio en la envolvente. Además, existe la posibilidad de que los valores de los mismos umbrales puedan estar generando un retardo en el tiempo de inicio de la estimulación eléctrica, ya que habrá movimientos incompletos que no logren pasar ese umbral, y por lo tanto no activar la estimulación eléctrica; una buena idea sería probar disminuir los umbrales para lograr la activación de la estimulación con movimientos ligeros, posiblemente acompañados de otros algoritmos de procesamiento. Otro aspecto importante de este clasificador, es la imposibilidad de compensar la fatiga muscular. Implementar un clasificador robusto basado en algún algoritmo de inteligencia artificial podría ser de mayor utilidad para una aplicación que considere estos y otros aspectos relevantes para su uso en rehabilitación.

En cuanto al mapeo proporcional encargado de la modulación de la amplitud FES, hay que destacar que este se obtiene a partir de dos umbrales obtenidos en la etapa de calibración, y considerando que el sEMG no presenta un comportamiento lineal, es probable que este método para la obtención de la ecuación del mapeo proporcional no pueda realizar un seguimiento preciso a los cambios existentes en las señales de sEMG. Una mejora a esta parte podría ser la implementación de un procedimiento de calibración con más de dos puntos, y a partir de ellos obtener una regresión lineal, o en su momento introducir algún modelo de control que represente la relación existente entre las señales sEMG y la fuerza ejercida o el torque muscular. Cabe destacar que debido a que la técnica de retroalimentación utilizada para este proyecto es el biofeedback, no existe como tal una señal de retroalimentación como parte del mapeo proporcional, porque no se usa una diferencia entre un setpoit y el valor de una variable de salida. Para que este mapeo fuera realmente un control proporcional, de acuerdo a su definición, la señal de control debería ser una señal de error obtenida a partir de la diferencia entre la señal ideal y la señal real, por ejemplo, la señal de sEMG del miembro no afectado y la señal de sEMG del miembro afectado o el ángulo o rango de movimiento de la muñeca esperado y el medido. El agregar una señal de retroalimentación objetiva que pueda ser considerada dentro del sistema de control probablemente mejorará el desempeño.

Relacionado a las pruebas con la aplicación sEMG-FES contralateral (validación en línea con FES y tarea funcional asíncrona), el hecho de que se lograra llevar a cabo con éxito a pesar de las limitaciones ya mencionadas de los distintos bloques del sistema, es un indicador del potencial que tiene el enfoque propuesto en este proyecto, para el desarrollo de aplicaciones para terapia de rehabilitación, dando a los pacientes la posibilidad de realizar tareas comunes de la vida diaria que serían imposibles de realizar para ellos de otro modo. Se espera que con las correctas adecuaciones, el sistema diseñado en este proyecto pueda ser de utilidad en sujetos con hemiplejia. Adicionalmente, sería bueno realizar una variante de esta prueba donde se llevara a cabo el mismo procedimiento pero mediante un experimento cruzado (un sujeto controla mientras otro sujeto es estimulado), y junto con algún algoritmo de visión por computadora poder determinar el retardo existente en esta prueba, para así tener una métrica más sobre el desempeño del sistema. O bien, introduciendo una señal sintética similar al sEMG que permita la medición exacta de los tiempos de retardo.


\subsubsection*{Conclusiones}
La aplicación sEMG-FES diseñada en este trabajo logró cumplir el objetivo principal del mismo, y además, pese a las limitaciones del sistema, se logró demostrar que el enfoque utilizado en este trabajo puede ser de utilidad en aplicaciones diseñadas para terapia de rehabilitación.

Relacionado al protocolo de registro se concluye que este es susceptible a errores humanos, por lo cuál es recomendable utilizar otro protocolo o seguir guías clínicas como las ya mencionadas guías del SENIAM.

En cuanto al algoritmo para clasificación de movimientos, se destaca que el algoritmo diseñado basado en umbrales fue útil para verificar la viabilidad del sistema sEMG-FES desarrollado, aunque con un desempeño subóptimo. En el futuro se pueden implementar soluciones basadas en algoritmos de inteligencia artificial, que probablemente arrojen mejores resultados.

Por último, relacionado al mapeo sEMG-FES, se obtuvo una solución rápida y sencilla, la cuál se puede adaptar a diferentes tipos de señales con las que se cuente. Sin embargo, al tener el sEMG un comportamiento no lineal es probable que este este método no sea el idóneo para dicha señal, abriendo la posibilidad al uso de otros esquemas de control.
