
Al integrar los diferentes bloques de un sistema de estimulación eléctrica funcional dentro de una misma plataforma de software con herramientas de tiempo real, se logrará ejecutar la adquisición de señales, procesamiento y estimulación de manera paralela y en lazo cerrado, con un tiempo de latencia en la actualización de parámetros menor a 1 segundo.