%[Tamaño de letra 12, hoja tamaño carta, idioma español]{Documento tipo reporte}
\documentclass[12pt,letterpaper]{report}

%Interlineado del documento
\renewcommand{\baselinestretch}{1.15}

%Permite configurar espaciados de forma sencilla
\usepackage{setspace}  

%Configuración de márgenes
\usepackage[left=2.5cm,right=2.5cm,top=2.5cm,bottom=2.5cm]{geometry}

%Permite el manejo de caracteres especiales en el español
\usepackage[utf8]{inputenc}
\usepackage[spanish]{babel}

%Permite manejo de ecuaciones y símbolos matemáticos
\usepackage{amsmath}
\usepackage{amsfonts}
\usepackage{amssymb}

%Permite manejo de gráficos
\usepackage{graphicx}
\graphicspath{{../figures/}}%Path de figuras

%Permite agregar sub-figuras
\usepackage{caption}
\usepackage{subcaption}

%Permite colorear texto
\usepackage{color}

%Permite manejo de múltiples columnas dentro de un mismo documento
\usepackage{multicol}

%Hipervínculos
\usepackage{hyperref}
\hypersetup{
    colorlinks=true,
    linkcolor=black,
    urlcolor=blue,
    citecolor=black}
    
%Para contraer referencias
\usepackage{cite}

%Permite que la sección bibliografía/referencias se muestre en el índice    
\usepackage[nottoc]{tocbibind}

%Símbolo de valor absoluto
\providecommand{\abs}[1]{\lvert#1\rvert}
    
\begin{document}

%Portada del documento
\begin{titlepage}
\begin{center}

\begin{figure}
	\centering
	%Logo UAM
	\includegraphics[scale=0.6]{variacion6Izt.png}
	\hfill
	%Logo INR
	\includegraphics[scale=2]{INR.png}
\end{figure}

\hfill \break

\LARGE{Reporte de avances de proyecto terminal de ingeniería biomédica}

\vspace{1cm}

\rule{18cm}{0.5mm}\\
\LARGE{Aplicación de estimulación eléctrica funcional en lazo cerrado para el control contralateral de la pinza gruesa de la mano\\}
\rule{18cm}{0.5mm}

\vspace{1cm}

\Large{	\textbf{Alumno}: Enrique Mena Camilo\\
		\textbf{Matrícula}: 2153009451\\}

\vspace{1cm}

\Large{\textbf{Asesores}:\\Dr. Omar Piña Ramirez\\M.C. Jorge Airy Mercado Gutierrez}

\vspace{1cm}

\Large{7 de Enero de 2020}

\end{center}
\end{titlepage}

%INDÍCE
\pagenumbering{roman}
\tableofcontents
\newpage
\pagenumbering{arabic}

\chapter{Introducción}
%INTRODUCCIÓN

En la división de Investigación en Ingeniería Médica del Instituto Nacional de Rehabilitación ``Luis Guillermo Ibarra Ibarra'' (INR) se llevan a cabo diversos proyectos de investigación y desarrollo tecnológico, los cuales buscan desarrollar tecnología que permita aplicar nuevas técnicas de terapia de rehabilitación para personas con discapacidad motriz, o bien mejorar las técnicas ya existentes. Tal es el caso del proyecto de desarrollo de una neuroprótesis basada en estimulación eléctrica funcional (FES, por sus siglas en inglés) para rehabilitación de miembro superior, que busca satisfacer las necesidades del INR en cuestiones de terapia e investigación.

Para dicho proyecto el INR desarrolló una interfaz gráfica de usuario (GUI, por sus siglas en inglés) que permite controlar los parámetros de los dispositivos comerciales RehaMove 2 (Hasomed GmbH, Alemania) para estimulación eléctrica, y Cyton Board (OpeBCI Inc, E.E.U.U.) para adquisición de biopotenciales, que la neuroprótesis necesita para su correcto funcionamiento. Una primera aplicación de la neuroprótesis se encuentra funcionando y se utilizado con sujetos sanos y pacientes del INR, aplicándoles estimulación eléctrica para generar movimientos de miembro superior. Sin embargo, actualmente el sistema opera en lazo abierto, es decir, se determina la secuencia de estimulación sin considerar información relevante como el movimiento generado y sus variables relacionadas. En el estado actual del sistema, los parámetros de la estimulación eléctrica son controlados por el experimentador, quien los adapta hasta obtener el patrón específico útil para el sujeto en rehabilitación, esto causa una dependencia del experimentador para la operación del sistema, y limita su utilidad para objetivos de rehabilitación, ya que no se consideran variables intrínsecas del paciente, como lo son: la intención de movimiento o actividad residual, que podían contribuir a la modulación de los parámetros de estimulación y el movimiento resultante.

%Adicional a la GUI desarrollada en el INR, incluyendo la implementación de un funcionamiento en lazo abierto, se desarrolló un prototipo de adquisición de biopotenciales que permitiera tener un control total sobre los parámetros de adquisición (frecuencia de muestreo, ganancia y filtros) y que además fuera compatible con la aplicación en lazo abierto antes mencionada.

\section{Justificación}
%{\color{blue}JUSTIFICACIÓN\\}

Debido a que la neuroprótesis desarrollada en el INR se ha utilizado para trabajar en aplicaciones donde los sistemas involucrados para su funcionamiento trabajan sin tener interacción entre ellos, surge la necesidad de implementar alguna aplicación que permita la interacción entre sistemas y que además permita la participación del paciente de forma cuantitativa en la terapia.

Por ello, este proyecto plantea desarrollar una aplicación en lazo cerrado que involucre la actividad voluntaria del paciente para lograr la modulación de la estimulación eléctrica y esta a su vez permita la repetición de los movimientos relacionados al agarre de un objeto (flexión y extensión de los dedos), dotándolo así de control sobre los movimientos en su rehabilitación y disminuyendo la dependencia del experimentador al momento de realizar la terapia de estimulación.

\section{Planteamiento del problema}
%{\color{blue}PROBLEMA\\}

Para realizar el diseño de la aplicación se utilizarán dos canales de electromiografía de superficie (sEMG, por sus siglas en inglés) para la extracción de algún rasgo descriptivo de su amplitud, el cuál servirá para realizar el diseño de un esquema de control que permita identificar el tipo de movimiento a realizar (flexión o extensión de los dedos) y a su vez sirva como modulador de la amplitud de corriente eléctrica a inyectar en el miembro contrario, buscando lograr un funcionamiento en tiempo real.

\section{Hipótesis}
%{\color{blue}HIPÓTESIS\\}

Por lo tanto, la hipótesis de este proyecto consiste en que al integrar los diferentes bloques de un sistema de estimulación eléctrica funcional dentro de una plataforma de software con herramientas de tiempo real, se logrará implementar un lazo cerrado que permita llevar a cabo las tareas de adquisición de sEMG, procesamiento de este y aplicación de estimulación eléctrica en línea.

\section{Objetivos}
%{\color{blue}OBJETIVOS\\}

Con esto, el objetivo de este proyecto se centra en diseñar e implementar un esquema de control que permita la modulación de la amplitud de estimulación eléctrica en tiempo real.

Teniendo como objetivos particulares los siguientes:

\begin{itemize}
	\item Desarrollar un bloque de adquisición dentro de Simulink que permita la recepción de datos desde la computadora.
	\item Evaluar la calidad del funcionamiento del bloque de adquisición diseñado.
	\item Desarrollar un algoritmo que permita la identificación de los movimientos de flexión y extensión de dedos.
	\item Diseñar e implementar un esquema de control que permita la modulación de la amplitud de la estimulación eléctrica a través de señales de sEMG.
\end{itemize}


\chapter{Marco teórico}

\section{Estimulación eléctrica funcional}
La estimulación eléctrica funcional es la aplicación de corriente eléctrica a tejido excitable para suplementar o reemplazar funciones que se han perdido en individuos con daños neurológicos. El propósito de la intervención FES es habilitar funciones que se han perdido en individuos con daño al sistema nervioso mediante la sustitución o asistencia a las habilidades voluntarias de dichos individuos. En las aplicaciones FES la estimulación es requerida para lograr una función deseada, por lo tanto, los sistemas FES usualmente se diseñan para ser controlados a partir de señales relacionadas a la actividad o intención del propio usuario. Los dispositivos FES que son usados para sustituir una función neurológica que se ha perdido son comúnmente llamadas neuroprótesis \cite{Peckham2005}.

\section{Neuroprótesis}
Una neuroprótesis (NP) es un dispositivo que proporciona ráfagas cortas de impulsos eléctricos al sistema nervioso central o periférico a través de electrodos superficiales, para lograr producir funciones sensoriales o motoras. Estos dispositivos buscan sustituir o asistir una función dañada debido a una lesión o enfermedad en el sistema nervioso \cite{Popovic2008}\cite{Popovic2015}.

En general, existen dos tipos de neuroprótesis: a) las neuroprótesis autónomas, las cuales son sistemas autocontenidos que imitan las funciones de una contraparte biológica, y b) las neuroprótesis por comando, las cuales son sistemas que reemplazan o asisten una función sensitiva o motora que se ha perdido o disminuido. Estas últimas están compuestos por un sistema de control que interpreta la intención del usuario, utilizan sensores para detectar el estado del sistema, genera la activación del sistema motor o sensorial del usuario, y proporciona una retroalimentación al usuario \cite{Popovic2015}.

Las neuroprótesis motoras, las cuales son un ejemplo de NP por comando, son sistemas que asisten a personas que han sufrido algún tipo de lesión en la médula espinal o cerebro. Estas NP pueden actuar directamente en el sistema nervioso central, en el sistema nervioso periférico o bien en una combinación de ambos \cite{Popovic2015}.

\section{Señales de comando y retroalimentación}
Como se muestra en la Figura \ref{Figura: CompNeuroP}, una neuroprótesis por comando requiere de dos señales esenciales para lograr su correcto funcionamiento, una de estas es una señal de comando y otra es una señal de retroalimentación \cite{Popovic2015}.

\begin{figure}[htbp]
\centering
	\includegraphics[scale=0.8]{CompNeuroP_ESP.png}
	\caption{Esquema de los componentes generales de una neuroprótesis autónoma y por comando. Adaptado de \cite{Popovic2015}.}
	\label{Figura: CompNeuroP}
\end{figure}

\subsection{Señal de comando}
Son señales utilizadas como indicadores de eventos de determinada tarea. En el caso de las neuroprótesis son las señales que controlan las acciones de esta, especialmente las acciones relacionadas a la estimulación eléctrica (inicio, fin, incremento de intensidad, disminución de intensidad, etc.).

\subsection{Señal de retroalimentación}
Es un tipo de señal que brinda al sistema información relacionada a la respuesta a un determinado comando. Estas señales, en el caso de las neuroprótesis, suelen estar relacionadas con el monitoreo del movimiento que está realizando el sujeto debido a los efectos de la estimulación eléctrica y pueden registrarse mediante distintos tipos de sensores.

\section{Esquemas de control}
Existen dos tipos de control importantes dentro de las aplicaciones de una neuroprótesis, los cuales se diferencian esencialmente en los tipos de señales que ocupan. En la Figura \ref{Figura: EsqCont} se ilustran a grandes rasgos las diferencias entre ambos esquemas de control \cite{Wright2016}.

\begin{figure}[htbp]
\centering
	\includegraphics[scale=0.8]{EsquemasControl_ESP.png}
	\caption{Esquema general de control en lazo abierto y control en lazo cerrado. El control en lazo abierto se ilustra con una línea sólida. El control en lazo cerrado se lleva a cabo cuando se incluye el elemento sensor, el cual se ilustra con una línea discontinua. Adaptado de \cite{Wright2016}.}
	\label{Figura: EsqCont}
\end{figure}

\subsection{Control en lazo abierto}
En el control en lazo abierto se genera un comando a la línea de base, esperando que este comando produzca la salida correcta. Aquí no existe una medición de la salida generada, por lo cual tampoco existe alguna medición del error que pudiera utilizarse como mecanismo para la modulación del comando que se genera \cite{Wright2016}.

\subsection{Control en lazo cerrado}
El control en lazo cerrado requiere de la inclusión de algún elemento sensor en el sistema que se desea controlar. Este control retroalimentado genera un comando a la línea de base y el elemento sensor mide la salida de la planta en respuesta al comando. Esta medición de la salida puede utilizarse para determinar diferencias entre la salida esperada y la real, generando así una señal de error que puede utilizarse como retroalimentación hacia el controlador para realizar modificaciones en los comandos generados \cite{Wright2016}.

Dichos esquemas de control suelen utilizar algunas de las siguientes políticas de control:
\begin{itemize}
	\item Control bang-bang (control On-Off): es una política de control en la que cuando una variable cruza un umbral predefinido, se activa un programa que habilita o deshabilita determinadas funciones del esquema de control \cite{Wright2016}.
	\item Máquina de estados finitos: es un modelo de sistema que puede considerarse como una implementación más compleja de la política On-Off. En este modelo, la medición de una variable del sistema en combinación con el estado actual desencadena una serie de acciones y una transición de estado. Este tipo de modelo es periódico, entonces pueden realizarse transiciones de estado en respuesta al tiempo \cite{Wright2016}.
\end{itemize}

{\color{red}AGREGAR CONTROLES PID}\\
{\color{blue}Enrique: ¿debería agregar algo sobre la conversión de números binarios en complemento 2?}

%\section{\color{blue}Números binarios}


%\subsection{Complemento a uno}


%\subsection{Complemento a dos}


%\subsection{Conversión complemento a dos a binario}


\chapter{Antecedentes}
\section{Desarrollos previos al proyecto}
{\color{red}Peña: ¿Dónde están publicados los trabajos realizados en el INR?}

En el INR se han realizado trabajos previos relacionados al desarrollo de la neuroprótesis, los cuales han logrado que dicho sistema sea funcional y se pueda ocupar en pacientes del propio instituto. Este trabajo incluye una plataforma de software para control y configuración de la neuroprótesis, la implementación de una aplicación FES en lazo abierto comandada por EEG, y un sistema prototipo de adquisición de biopotenciales.

\subsection{Plataforma de software para neuroprótesis}
Consiste en una GUI implementada en la herramienta GUIDE de MATLAB®, la cual consta de 4 pantallas que en conjunto permiten, hasta el momento: a) realizar el registro de datos de un paciente o usuario en el que se probará el dispositivo, b) realizar el entrenamiento de un clasificador de movimientos voluntarios, c) ejecutar una aplicación FES en lazo abierto, o bien d) experimentar con los parámetros del estimulador y el sistema de registro de biopotenciales para determinar el patrón de estimulación óptimo para el paciente. Esta plataforma realiza una conexión a dispositivos comerciales: Rehastim 2 (Hasomed GmbH, Alemania) para estimulación eléctrica, y Cyton Board (OpeBCI Inc, E.E.U.U.) para adquisición de biopotenciales) que permiten la integración de las funciones de la NP.

\subsection{Aplicación FES en lazo abierto}
La aplicación FES, que se encuentra inmersa en la plataforma de software para la neuroprótesis, está basada en una Interfaz Cerebro-Computadora. Dicha aplicación le muestra al sujeto una serie de 5 movimientos predefinidos, dentro de los cuales el sujeto debe seleccionar alguno cerrando los ojos. Una vez seleccionado y confirmado el movimiento objetivo, el sistema envía una secuencia de pulsos de estimulación eléctrica para asistir al sujeto a realizar el movimiento elegido. En esta aplicación el patrón de estimulación eléctrica está predeterminado antes de iniciar la aplicación.

\subsection{Sistema prototipo de adquisición de biopotenciales}
Sistema que consta del convertidor analógico digital ADS1299 y el microcontrolador MSP432P401R. Es un sistema que presenta ventajas respecto al sistema comercial utilizado en trabajos anteriores basados en OpenBCI, principalmente una frecuencia de muestreo de 1 kHz por canal, la cual es útil para fines de control con sEMG \cite{Lenzi2012}\cite{Lenzi2011}\cite{Raafat}. Además, el prototipo utiliza una conexión USB para la transmisión de datos, la cual, a diferencia de la conexión bluetooth con la que cuenta el dispositivo de OpenBCI, permite una mayor tasa de transmisión de datos (460800 bps, contra 115200 bps). %y evita la pérdida de datos que se presentaba en el sistema comercial por fallas en la conexión bluetooth.

El sistema prototipo de adquisición será útil para fines de este proyecto, ya que gracias a la interfaz gráfica desarrollada previamente para dicho sistema, se tiene la posibilidad de ajustar los parámetros de adquisición de tal forma que nos permitirá obtener una señal de sEMG de mejor calidad que con el sistema OpenBCI.

\section{Sistemas FES existentes}
En el Cuadro ~\ref{Cuadro:Sistemas FES} se muestran los trabajos revisados que proporcionan información de interés para lograr los objetivos de este proyecto. Dentro de los campos que se destacan de dichos trabajos se encuentran: la aplicación, debido a que se buscaron trabajos que asistan el funcionamiento de las extremidades, en especial de miembro superior; el dispositivo de estimulación, ya que se buscaron trabajos que utilizaran el mismo dispositivo a utilizar en este proyecto o bien sus versiones anteriores; la implementación del esquema de control, esto debido a que se buscaron sistemas que aprovecharan el entorno de Simulink, ya que el controlador del dispositivo de estimulación eléctrica a emplear (Rehastim2) está desarrollado en dicha plataforma; y finalmente, las señales que dichos sistemas utilizaron para realizar la retroalimentación del sistema y la activación de los comandos.

De estos trabajos se puede rescatar que, al realizar un entrenamiento en espejo donde sea un miembro sano el que controla la estimulación eléctrica aplicada al miembro dañado, se lograrán disminuir los artefactos generados por esta al momento de registrar EMG, o bien serán nulos si se ocupa una técnica de cuantificación de movimiento de origen no bioeléctrico \cite{Salchow2016}. También, se destaca que para realizar una terapia de asistencia para apertura y cierre de mano es necesario tener indicadores del estado actual de la mano y del estado del objeto sobre el que se quiere realizar la acción \cite{Simonsen2017}. Adicional a esto, se ha demostrado que implementar una máquina de estados finitos para el control de una neuroprótesis es algo viable y que permite la comprensión rápida, por parte del usuario, del funcionamiento del esquema de control \cite{Sun2014}. Por último, se destaca que, de lograr integrar todos los componentes del esquema de control en una misma plataforma, se pueden realizar aplicaciones que presenten un funcionamiento en tiempo real o muy cercano a este \cite{Salchow2016}\cite{Sun2014}\cite{Woods2018}.

%Cuadro de revisión bibliográfica
\begin{table}[hbt]
	\centering
	\begin{tabular}{|p{25mm}|p{35mm}|p{25mm}|p{40mm}|p{35mm}|}
	\hline
	\textbf{Referencia} & \textbf{Aplicación} & \textbf{Dispositivo de estimulación} & \textbf{Señales de comando y retroalimentación} & \textbf{Implementación del sistema de control}\\ 
	\hline
	%\cite{Salchow2016}
	(Salchow, 2016) & Entrenamiento en espejo para posturas de mano & RehaMove Pro & Electromiografía, movimiento de mano & MATLAB/Simulink\\
	\hline
	%\cite{Sun2014}
	(Sun, 2014) & Recuperación de funciones de miembro superior & RehaStim1 & Acelerómetro & Simulink\\
	\hline
	%\cite{Simonsen2017}
	(Simonsen, 2017) & Asistencia para apertura y cierre de mano & STMISOLA & Posición del objeto, posición de la mano & MATLAB\\
	\hline
	%\cite{Woods2018}
	(Woods, 2018) & Asistencia en miembro inferior para ciclisco & RehaStim 1 & Mecanomiografía, fuerza aplicada a pedales, posición del cigüeñal & Simulink\\
	\hline
	\end{tabular}
	\centering
	\caption{Revisión de sistemas FES reportados en la literatura con aplicaciones similares a las de este proyecto.}
	\label{Cuadro:Sistemas FES}
\end{table}

{\color{red}Agregar revisión de métodos de procesamiento y control para aplicaciones de FES contralateral en otro cuadro}

\begin{table}[htb]
	\centering
	\begin{tabular}{|c|c|c|c|}
	\hline
	\textbf{Referencia} & \textbf{Aplicación} & \textbf{Rasgo utilizado} & \textbf{Tipo de control}
		
	
	
	\end{tabular}
\end{table}

\chapter{Metodología}
\section{Sistema propuesto}
Para este proyecto se planteó un sistema que implementa un control de FES en lazo cerrado utilizando la técnica de biofeedback. El sistema consiste en la adquisición de dos canales de sEMG, del brazo izquierdo, los cuales son procesados y se utilizan como entrada de un sistema de control que realiza la modulación de la amplitud de dos canales de estimulación eléctrica en el brazo derecho. Este sistema implementa un control contralateral para realizar un entrenamiento en espejo de las acciones de apertura y cierre de mano.

En la Figura \ref{Figura: SistProp} se muestra un esquema general del sistema desarrollado, en el cual se muestran en rojo los elementos implementados en este proyecto.

%Sistema propuesto
\begin{figure}[htbp]
\centering
	\includegraphics[scale=0.7]{SistemaPropuesto.png}
	\caption[Sistema propuesto para el proyecto]{Sistema propuesto para el proyecto. Líneas continuas representan entes de hardwares y líneas discontinuas representan entes de software. Elementos en rojo representan zonas de trabajo del proyecto.}
	\label{Figura: SistProp}
\end{figure}


\section{Adquisición de datos en Simulink}
Se utilizó el sistema Cyton Board, el cual tiene una frecuencia de muestreo de 250 Hz, para realizar la adquisición de las señales de sEMG. Dicho sistema utiliza un chip ADS1299 (Texas Instruments Inc., Dallas, E.U.A.) para realizar la conversión analógico-digital de las señales, el cual codifica los datos de cada muestra, utilizando complemento a 2, en un flujo de datos de 27 bytes esquematizado en la Figura \ref{Figura: BusOut}.

%Stream datos ADS
\begin{figure}[htbp]
\centering
	\includegraphics[scale=0.8]{Bus_Dat_Out_ADS.png}
	\caption{Flujo de datos de salida del ADS1299.}
	\label{Figura: BusOut}
\end{figure}

Para realizar la decodificación del flujo de datos dentro de Simulink, se diseñó un subsistema encargado de la solicitud y decodificación de datos, para esto se utilizó el bloque \emph{Query Instrument} del \emph{Instrument Control Toolbox} para realizar la solicitud de datos, los cuales fueron decodificados con bloques de la librería estándar de Simulink. La figura \ref{Figura: DecoSimuT} muestra la implementación dentro de Simulink del sistema antes mencionado.

\hfill \break

%Sistema Simulink
\begin{figure}[htbp]
	\centering
	\begin{subfigure}[htbp]{0.8\textwidth}
		\includegraphics[width=\textwidth]{Read_Simu.png}
		\caption{Vista general del subsistema diseñado para realizar adquisición y decodificación del flujo de datos.}
		\label{Figura: readSimu}
	\end{subfigure}
	\begin{subfigure}[htbp]{0.8\textwidth}
		\includegraphics[width=\textwidth]{Deco_Simu.png}
		\caption{Vista interna del subsistema encargado de la decodificación del flujo de datos.}
		\label{Figura: decoSimu}
	\end{subfigure}
	\caption[Subsistema decodificador del flujo de datos]{Subsistema decodificador del flujo de datos implementado en Simulink.}
	\label{Figura: DecoSimuT}
\end{figure}

\newpage
El funcionamiento del subsistema responsable de la solicitud y decodificación de datos implementado dentro de Simulink se encuentra esquematizado en la Figura \ref{Figura: DecoStream}, y lleva a cabo el siguiente algoritmo:

\begin{enumerate}
	\item Se realiza la adquisición de N muestras, lo cual generará un vector columna con dimensión $\mathbb{R}^{27*N\times 1}$ (Figura \ref{Figura: DecoStream} \emph{(a)}).
	\item Se aplicar un reshape a dicho vector para obtener una matriz con dimensión $\mathbb{R}^{27\times N}$ (Figura \ref{Figura: DecoStream} \emph{(b)}).
	\item Se obtiene la transpuesta de dicha matriz para obtener una matriz con dimension $\mathbb{R}^{N\times 27}$ (Figura \ref{Figura: DecoStream} \emph{(c)}).
	\item Se realiza la extracción de las columnas asociadas al canal a procesar, obteniendo una matriz con dimensión $\mathbb{R}^{N\times 3}$ (Figura \ref{Figura: DecoStream} \emph{(d)}).
	\item Se realiza el producto matricial de la matriz del canal a procesar con un vector ponderador que contiene el peso de cada columna (byte) para el valor de la muestra de 24 bits (Figura \ref{Figura: DecoStream} \emph{(e)}). El vector ponderador está compuesto por los valores $2^{16}$, $2^8$, $1$. Como resultado de dicho producto se obtiene un vector de dimensión $\mathbb{R}^{N\times 1}$ (Figura \ref{Figura: DecoStream} \emph{(f)}).
	\item Se extraen del vector anterior las muestras en las que se encuentren codificados, en complemento a 2, un número negativo. Esto se realiza al obtener el valor del bit 23 (bit más significativo), y si dicho bit tiene un valor de 1 implica que dicha muestra codifica un número negativo (Figura \ref{Figura: DecoStream} \emph{(g)}).
	\item Se obtiene el complemento a 1 de cada muestra del vector \emph{g}, se suma 1 a cada muestra y se multiplica cada muestra por -1. De esta operación se obtiene un vector con las muestras decodificadas a número negativos (Figura \ref{Figura: DecoStream} \emph{(h)}).
	\item Se insertan los elementos del vector \emph{h} en sus posiciones originales del vector \emph{f}. De este último paso se obtiene un vector con las N muestras decodificadas a números de 24 bits con signo (Figura \ref{Figura: DecoStream} \emph{(i)}).
\end{enumerate}

%Funcionamiento decodificación
\begin{figure}[htbp]
\centering
	\includegraphics[width=\textwidth]{DecoStream.png}
	\caption{Funcionamiento del subsistema decodificador del flujo de datos.}
	\label{Figura: DecoStream}
\end{figure}


\newpage
\section{Evaluación de bloque de adquisición y decodificación}
Se llevó a cabo un procedimiento para evaluar el desempeño del subsistema diseñado en Simulink para la adquisición y decodificación de datos. En MATLAB se generó un banco de señales senoidales conformado por 5 senoidales puras de 1 Hz, 5 Hz, 10 Hz, 20 Hz y 50 Hz (\ref{Figura: SenPur}), dos senoidales de 50 Hz moduladas en amplitud con una envolvente de una recta con pendiente negativa (\ref{Figura: LinAte}) y una exponencial decreciente (\ref{Figura: ExpAte}), y una senoidal de 50 Hz modulada en amplitud con una envolvente que simula la señal sEMG correspondiente a la tarea de incrementar gradualmente una contracción muscular, mantener dicha contracción y relajar el músculo gradualmente (\ref{Figura: Contra}). Todas las señales del banco se diseñaron con una frecuencia de muestreo de 250 Hz y una duración de 5 segundos, excepto la última, que se diseñó con una duración de 15 segundos; adicionalmente, todas las señales se generaron como objetos de audio dentro de MATLAB, para poder reproducirlas como audio y a partir de la salida de audio de la computadora poder acceder a ellas.

%Figura de senoidales MATLAB
\begin{figure}[htbp]
	\centering
	\begin{subfigure}[htbp]{0.5\textwidth}
		\includegraphics[width=\textwidth]{Sen_Pur.png}
		\caption{Senoidales puras a diferentes frecuencias.}
		\label{Figura: SenPur}
	\end{subfigure}
	\hfill
	\begin{subfigure}[htbp]{0.4	\textwidth}
		\includegraphics[width=\textwidth]{Lin_Ate.png}
		\caption{Senoidal de 50 Hz con atenuación lineal.}
		\label{Figura: LinAte}
	\end{subfigure}
	\hfill
	\begin{subfigure}[htbp]{0.4\textwidth}
		\includegraphics[width=\textwidth]{Exp_Ate.png}
		\caption{Senoidal de 50 Hz con atenuación exponencial.}
		\label{Figura: ExpAte}
	\end{subfigure}
	\hfill
	\begin{subfigure}[htbp]{0.4\textwidth}
		\includegraphics[width=\textwidth]{Contra.png}
		\caption{Senoidal de 50 Hz simulando el sEMG de una contracción muscular.}
		\label{Figura: Contra}
	\end{subfigure}	
	\caption{Banco de señales para evaluación de adquisición.}
	\label{Figura: SenalesEva}
\end{figure}

\newpage
El proceso para la evaluación del funcionamiento del subsistema de adquisición se llevó a cabo de la siguiente manera:
\begin{enumerate}
	\item Se realiza la adquisición de tres tantos de todas las señales del banco de señales de prueba.
	\begin{enumerate}
		\item Se conecta una punta de un jack de audio de 3.5 mm (Figura NUM) a la salida de audio de la computadora. La otra punta se conectó al dispositivo de adquisición (Cyton Board) de la siguiente forma: los pines \emph{izquierdo} y \emph{derecho} se conectaron a la entrada diferencial, mientras que el pin \emph{tierra} se conectó a la entrada BIAS.
		\item Se realiza la solicitud de datos utilizando el subsistema decodificador implementado en Simulink, y al mismo tiempo se inicia el conteo de un cronómetro.
		\item Tras haber transcurrido 2 s en el cronómetro, se procede a reproducir la señal de audio de prueba.
		\item Tras haber transcurrido 10 s (20 s para la simulación del sEMG de una contracción), se detiene la adquisición del subsistema de Simulink y se guardan los datos de adquisición dentro de un archivo con extensión \emph{.mat}.
	\end{enumerate}
	\item Se cargan, dentro del workspace de MATLAB, los datos de las señales adquiridas y los datos de las señales patrón.
	\item Se procede a calcular el coeficiente de correlación de Pearson (Ecuación \ref{Ecu: CorrePea}) entre las señales adquiridas y su correspondiente señal patrón.
	\item Se obtiene la media aritmética de todos los valores obtenidos al aplicar el coeficiente de correlación de Pearson. El valor obtenido se utiliza como indicador de la calidad del subsistema de adquisición y decodificación de datos.
\end{enumerate}

%Ecuación correlación Pearson
\begin{equation}
	r = \frac{\sigma_{xy}}{\sigma_{x}\sigma_{y}}
	\label{Ecu: CorrePea}
\end{equation}


\section{Protocolo para registro de sEMG}
Para garantizar una repetibilidad en los registros de sEMG se implementó un protocolo para realizar la adquisición de dicha señal. Dicho protocolo tiene las caracteristicas mostradas a continuación:

\begin{itemize}
	\item Frecuencia de muestro de sistema de adquisición: 250 Hz.
	\item Canales 1 y 2 para realizar adquisición.
	\item Electrodos: Covidien H124SG
	\item Canal 1: Digitorum flexor
	\begin{itemize}
		\item Medir antebrazo de lado ventral de codo a muñeca.
		\item Palpar músculo al 30\% de la medida obtenida.
		\item Colocar dos electrodos separados 2 cm (Figura \ref{Figura: E_Cie}).
	\end{itemize}
	\item Canal 2: Digitorum extensor
	\begin{itemize}
		\item Medir antebrazo de lado dorsal de codo a muñeca.
		\item Palpar músculo al 50\% de la medida obtenida.
		\item Colocar dos electrodos separados 2 cm (Figura \ref{Figura: E_Ape}).
	\end{itemize}
	\item Referencia: Colocar electrodo en codo.
\end{itemize}

\begin{figure}[htbp]
	\centering
	\begin{subfigure}[htbp]{0.3\textwidth}
		\includegraphics[width=\textwidth]{E_Cie.png}
		\caption{Ubicación de electrodos para digitorum flexor.}
		\label{Figura: E_Cie}
	\end{subfigure}
	\hfill
	\begin{subfigure}[htbp]{0.3\textwidth}
		\includegraphics[width=\textwidth]{E_Ape.png}
		\caption{Ubicación de electrodos para digitorum extensor.}
		\label{Figura: E_Ape}
	\end{subfigure}
	\caption[Posicionamiento de electrodos para registro de sEMG]{Posicionamiento de electrodos para realizar registros de sEMG. Recuperado de \cite{Cavalcanti-Garcia2009}.}
	\label{Figura: E_sEMG}
\end{figure}

\section{Procesamiento de sEMG}

Se diseñaron tres filtros Butterworth para realizar el procesamiento de sEMG: un filtro pasa altas con frecuencia de corte de 15 Hz,para eliminar las variaciones en la línea base del registro; un filtro pasa bajas con frecuencia de corte de 100 Hz, para eliminar armónicos de 60 Hz y demás interferencias de alta frecuencia; y un filtro rechaza banda centrado en 60 Hz, para reducir la interferencia de la línea. Las gráficas de respuesta en frecuencia de estos filtros se muestran en las Figuras \ref{Figura: FiltroPA} a \ref{Figura: FiltroRB}.

\begin{figure}[htbp]
	\centering
	\includegraphics[width=0.7\textwidth]{FiltroPA15Hz_PADQ.png}
	\caption{Filtro pasa altas para conseguir línea base estable.}
	\label{Figura: FiltroPA}
	
	\includegraphics[width=0.7\textwidth]{FiltroPB100Hz_PADQ.png}
	\caption{Filtro pasa bajas para eliminar interferencias de alta frecuencia y armónicos de 60 Hz.} 
	\label{Figura: FiltroPB}
	
	\includegraphics[width=0.7\textwidth]{FiltroRB_58_62_PADQ.png}
	\caption{Filtro rechaza banda para reducir interferencia de 60 Hz.}
	\label{Figura: FiltroRB}
\end{figure}

Se implementó dentro de Simulink un bloque responsable de obtener el valor RMS de ventanas de registro de 100 ms de sEMG para utilizar dicho descriptor de amplitud como señal de control. Adicionalmente se implementó un filtro de de mediana de 10 muestras (Ecuación \ref{Ecu: Mediana}), el cual tiene como propósito conseguir una señal de RMS suavizada

%Ecuación filtro mediana
\begin{equation}
	y[n] = mediana(x[n]:x[n-N])
	\label{Ecu: Mediana}
\end{equation}

\newpage
\section{Esquema de control}
Se diseñó un sistema basado en una combinación de máquina de estados finitos con un control lineal. El sistema requiere de un proceso de calibración previa donde se obtienen 8 umbrales tras la repetición de 4 movimientos, dos umbrales corresponden a los valores RMS promedio de los dos canales de adquisición a lo largo de la tarea \emph{cierre de mano ligero}, otros dos corresponden a los valores RMS promedio de la tarea \emph{cierre de mano completo}, mientras que los 4 restantes corresponden a los valores RMS promedio de las tareas \emph{apertura de mano ligera} y \emph{apertura de mano completa}. Además, tras la calibración se obtiene también un factor denominado \emph{detector de movimiento}, el cual se obtiene tras calcular la diferencia promedio entre los canales de adquisición a lo largo de la tarea de apertura de mano. Adicionalmente se realiza una calibración de la estimulación eléctrica, la cual utiliza el sistema de colocación de electrodos de estimulación descrito en \cite{AnaMartin2019}, donde se obtiene los valores en amplitud de los umbrales motores y funcionales de las tareas de apertura y cierre de mano.

El detector de movimiento se utiliza para realizar el control por máquina de estados finitos (Figura \ref{Figura: FSM_control}), la cual consisten en determinar si la diferencia de amplitudes entre canales ha pasado el valor del detector de movimiento, si es así, el control prosigue con la tarea de apertura de mano, en caso contrario, el control procede a la tarea de cierre de mano.

\begin{figure}[htbp]
	\centering
	\includegraphics[scale=0.9]{FSM_Control.png}
	\caption{Máquina de estados finitos encargada de la detección de movimiento e inicio del control lineal.}
	\label{Figura: FSM_Control}
\end{figure}

Dentro del control de cada tarea, se utilizan los umbrales de las tareas ligeras para realizar la activación del control lineal, el cual modula la amplitud de la corriente eléctrica, del canal asociado al movimiento detectado, utilizando la recta descrita por la Ecuación \ref{Ecu: Mapeo}, donde $A$ representa la amplitud que inyectará el estimulador eléctrico, $A_{max}$ es el umbral funcional de estimulación eléctica, $A_{min}$ es el umbral motor de estimulación eléctica, $D$ representa el valor RMS actual, mientras que $D_{max}$ y $D_{min}$ representan los umbrales RMS de la tarea completa y ligera del canal asociado al movimiento detectado (canal 1 para cierre de mano y canal 2 para apertura de mano). Adicionalmente se aplica la función máximo entero a la recta debido a que el dispositivo de estimulación eléctrica sólo admite valores enteros, y también se aplica un criterio de saturación de corriente eléctrica para evitar que tras una contracción muscular muy fuerte se genere un valor de amplitud de corriente eléctrica dañino para el sujeto.

%Ecuación mapeo lineal
\begin{equation}
	A = \frac{A_{max} - A_{min}}{D_{max} - D_{min}}(D - D_{min}) + A_{min}
	\label{Ecu: Mapeo}
\end{equation}

%\section{Tarea objetivo}
%{\color{red}INCLUIR TAREA OBJETIVO DE ENTRENAMIENTO CON TRAPEZOIDAL EN LÍNEA}


\chapter{Resultados}
%%RESULTADOS

\section{Evaluación de bloque de adquisición y decodificación}
Tras realizar la evaluación del subsistema de adquisición descrito en la sección \ref{Sec: Adquisicion} utilizando el procedimiento detallado en la sección \ref{Sec: EvalAdquisicion}, se obtuvo un valor de correlación promedio de 0.9615 $\pm$ 0.0604, el cual se obtuvo de un total de 27 registros realizados (3 repeticiones de cada una de las señales que conforman el banco de señales para evaluación de la adquisición). En la Figura \ref{Figura: ValProCum} se puede observar una comparación entre la señal patrón de 5 Hz y la señal adquirida con el subsistema diseñado en Simulink\textregistered. El Cuadro \ref{Cuadro:ValoresCorre} muestra el valor de correlación promedio obtenido para cada señal y la correlación total.

%Cuadro valores correlación
\begin{table}[htbp]
	\centering
	\begin{tabular}{|p{6cm}|l|}
	\hline
	\textbf{Señal} & \textbf{Correlación promedio}\\ \hline	\hline
	1 Hz & 0.9889\\ \hline
	5 Hz & 0.9428\\ \hline
	10 Hz & 0.9948\\ \hline
	20 Hz & 0.9933\\ \hline
	50 Hz & 0.9804\\ \hline
	100 Hz & 0.9334\\ \hline
	Atenuación lineal & 0.9466\\ \hline
	Atenuación exponencial & 0.8849\\ \hline
	Simulación contracción muscular & 0.9886\\ \hline
	\textbf{Correlación promedio total} & 0.9615\\ \hline
	\end{tabular}
	\caption{Valores de correlación promedio por señal y correlación promedio total.}
	\label{Cuadro:ValoresCorre}
\end{table}

%Tras adquirir las señales patrón para la evaluación del bloque de adquisición descritas en la metodología, se calculó la métrica de correlación entre las señales adquiridas y las patrón, buscando traslapar una sobre otra como se muestra en la Figura \ref{Figura: ValProCum}. Al tener el valor de correlación para cada registro se obtuvo como resultado una correlación promedio de 0.9615 $\pm$ 0.0604, valor que sirve como indicador de la calidad del bloque diseñado para la adquisición y decodificación de datos.

%Senoidal obtenida tras para evaluación
\begin{figure}[htbp]
	\centering
	\includegraphics[width=\textwidth]{ValProCum.png}
	\caption[Comparación entre señales para evaluación de adquisición.]{Comparación entre señales para evaluación de adquisición. Señal patrón generada en MATLAB\textregistered \; (azul). Señal adquirida mediante el subsistema de adquisición diseñado en Simulink\textregistered \; (Rojo).}
	\label{Figura: ValProCum}
\end{figure}

\newpage
\section{Procesamiento de sEMG}
El esquema de filtrado utilizado (filtro pasa altas, filtro pasa bajas y filtro rechaza banda), al igual que el procesamiento para obtención del RMS suavizado, se pusieron a prueba fuera de línea con registros de 10 voluntarios sanos (6 hombres y 4 mujeres en el rango de 20 a 24 años de edad).

La Figura \ref{Figura: Filtrado} muestra una comparación entre los canales de sEMG adquiridos para las pruebas de procesamiento y el resultado del filtrado fuera de línea. En la Figura \ref{Figura: RMS} se muestra un ejemplo del resultado del procesamiento para obtención de la envolvente RMS suavizada.

%Utilizando los registros de calibración se probaron los filtros diseñados, obteniendo como resultado notorio la estabilización de la línea base de cada registro. En la Figura \ref{Figura: Filtrado} se muestra una comparación entre los registros crudos y filtrados de ambos canales adquiridos durante el entrenamiento.

\begin{figure}[htbp]
	\centering
	\begin{subfigure}[htbp]{0.45\textwidth}
		\includegraphics[width=\textwidth]{Filtrado_a.png}
		\caption{}
		\label{Figura: Filtrado_a}
	\end{subfigure}
	\hfill
	\begin{subfigure}[htbp]{0.45\textwidth}
		\includegraphics[width=\textwidth]{Filtrado_b.png}
		\caption{}
		\label{Figura: Filtrado_b}
	\end{subfigure}	
	\caption[Ejemplo representativo del funcionamiento del esquema de filtrado diseñado]{Ejemplo representativo del funcionamiento del esquema de filtrado diseñado.(a)Arriba, registro de sEMG del canal 1 sin filtrar.(a)Abajo, registro de sEMG del canal 1 después del filtrado.(b)Arriba, registro de sEMG del canal 2 sin filtrar.(b)Abajo, registro de sEMG del canal 2 después del filtrado.}
	\label{Figura: Filtrado}
\end{figure}

%Con los registros ya filtrados se obtuvo el valor RMS a lo largo de todo el registro utilizando ventanas de 100 ms, dando como resultado una envolvente discreta de sEMG para cada canal. En la Figura \ref{Figura: RMS} se muestran los registros de sEMG filtrados con sus respectivas envolventes discretas de RMS y marcadores de la acción solicitada al sujeto durante el entrenamiento.

\begin{figure}[htbp]
	\centering
	\includegraphics[width=\textwidth]{RMS.png}
	\caption[Ejemplo representativo de la obtención de envolvente de RMS]{Ejemplo representativo de la obtención de envolvente de RMS. En azul, los registros de sEMG después del filtrado. En rojo, las envolventes de RMS. Arriba, señal sEMG y envolvente del canal 1, correspondiente al movimiento de pinza gruesa. Abajo, señal sEMG y envolvente del canal 2, correspondiente al movimiento de apertura de mano.}
	\label{Figura: RMS}
\end{figure}


\newpage
\section{Sistema de control}
El sistema de control sEMG-FES se puso a prueba con un voluntario sano de 22 años de edad, obteniendo los siguientes resultados.

\subsection{Calibración}

\subsection{Validación fuera de línea}
El algoritmo de clasificación de movimientos obtuvo una exactitud del 81\%. La Figura \ref{Figura: MapOff} muestra el resultado de la prueba para la validación fuera de línea, donde se pueden observar los siguientes resultados de clasificación:

\begin{itemize}
	\item Durante los episodios del movimiento de cierre de mano (9-21 y 51-63 s):
	
	\begin{itemize}
		\item Se puede observar una clasificación correcta cuando la señal \emph{Acción} toma como valores a \emph{CI} y \emph{CC} y la señal \emph{Amplitud FES$_{C1}$} toma valores distintos de cero mientras que la señal \emph{Amplitud FES$_{C2}$} toma valor de cero.
		\item Dentro de los episodios de este movimiento se pueden observar errores de clasificación en los segundos 18-21 y alrededor de los 59 segundos, donde se observa que la señal \emph{Amplitud FES$_{C2}$} toma valores distintos de cero.
	\end{itemize}
	
	\item Durante los episodios del movimiento de apertura de mano (30-42 y 72-84 s):
	
	\begin{itemize}
		\item Se puede observar una clasificación correcta cuando la señal \emph{Acción} toma como valores a \emph{AI} y \emph{AC} y la señal \emph{Amplitud FES$_{C2}$} toma valores distintos de cero mientras que la señal \emph{Amplitud FES$_{C1}$} toma valor de cero.
		\item Dentro de los episodios de este movimiento se pueden observar errores de clasificación en los segundos 30-33, 40-42 y 72-75, donde se observa que la señal \emph{Amplitud FES$_{C1}$} toma valores distintos de cero.
	\end{itemize}
		
	\item Durante los episodios de descanso (0-9, 21-30, 42-51, 63-72 y 84-87 s):
	\begin{itemize}
		\item Se puede observar una clasificación correcta cuando la señal \emph{Acción} toma como valor a \emph{DD} y las señales \emph{Amplitud FES$_{C1}$} y \emph{Amplitud FES$_{C2}$} toman valor de cero.
		\item Dentro de los episodios de este movimiento se pueden observar errores alrededor de los segundos 42 y 63, donde la señal \emph{Amplitud FES$_{C1}$} se activa brevemente cuando no tendría que haberse activado.
	\end{itemize}
\end{itemize}

%Figura validación fuera de línea
\begin{figure}[htbp]
	\centering
	\includegraphics[width=\textwidth]{MapOff.png}
	\caption[Secuencia temporal de una prueba exitosa de validación fuera de línea]{Secuencia temporal de una prueba exitosa de validación fura de línea del sistema de control sEMG-FES. Arriba: Envolventes de sEMG (Azul: canal 1. Rojo: canal 2). Centro: Amplitudes de estimulación eléctrica (salida del sistema de control) (Azul: canal 1. Rojo: canal 2). Abajo: Marcadores de acción solicitada al sujeto (descanso (DD), pinza gruesa incompleta (CI), pinza gruesa completa (CC), apertura incompleta (AL), apertura completa (AC)).}
	\label{Figura: MapOff}
\end{figure}


\newpage
\subsection{Validación en línea (control por biofeedback)}
Respecto a la prueba de validación en línea, en esta demostró una respuesta del sistema de acuerdo a lo esperado. La Figura \ref{Figura: MapOn} presenta un segmento de las señales obtenidas tras la realización de la prueba en línea.

%Figura prueba en línea (sin estimulación)
\begin{figure}[htbp]
	\centering
	\includegraphics[width=\textwidth]{MapOn.png}
	\caption[Secuencia temporal de una prueba exitosa de validación en línea]{Secuencia temporal de una prueba exitosa de validación en línea del sistema de control sEMG-FES.  Arriba: Envolventes de sEMG (Azul: canal 1. Rojo: canal 2). Centro: Amplitudes de estimulación eléctrica (salida del sistema de control) (Azul: canal 1. Rojo: canal 2). Abajo: Señal trapezoidal patrón indicadora del movimiento objetivo (descanso (DD), pinza gruesa (CC), apertura completa (AC)).}
	\label{Figura: MapOn}
\end{figure}


\newpage
\subsection{Tarea funcional síncrona}
En relación a la tarea funcional síncrona, el sistema logró llevar a cabo la modulación de la estimulación eléctrica de forma satisfactoria, obteniendo un retardo promedio del sistema con valor de de 2.3 $\pm$ 0.3553 s, medido de la forma descrita en la sección \ref{Sec: TareaObj}. La Figura \ref{Figura: Retardo} muestra un acercamiento a las señales obtenidas al termino de la prueba de la tarea funcional síncrona, donde es notable el retardo entre la señal patrón y la señal de amplitud de estimulación eléctrica.

%Figura medición de retardo
\begin{figure}[htbp]
	\centering
	\includegraphics[width=\textwidth]{Retardo.png}
	\caption[Secuencia temporal de una prueba exitosa de la tarea funcional síncrona]{Secuencia temporal de una prueba exitosa de la tarea funcional síncrona. Se muestran las diferentes señales asociadas a cada movimiento sobre la misma base de tiempo para visualizar el retardo existente entre la señal trapezoidal patrón y la señal de amplitud de estimulación eléctrica. Arriba: Señales para movimiento pinza gruesa completa (CC). Abajo: Señales para movimiento apertura completa (AC). En azul se muestra la señal trapezoidal patrón del movimiento objetivo. En rojo se muestra la envolvente de sEMG. En amarillo se muestra la amplitud de estimulación eléctrica (salida del sistema control).}
	\label{Figura: Retardo}
\end{figure}


\newpage
\subsection{Tarea funcional asíncrona}
Respecto a la tarea funcional asíncrona, esta logró ser realizada por el sujeto de prueba, logrando realizar las 6 acciones de la que consta dicha tarea. La Figura \ref{Figura: TareaFuncional} muestra las diferentes fases de movimiento durante la ejecución de la tarea funcional asíncrona por parte del sujeto. Se puede apreciar que los movimientos corresponden a los descritos en la sección \ref{Sec: TareaFunAsin}.

%Figura levantar objetos
\begin{figure}[htbp]
	\centering
	\begin{subfigure}[htbp]{0.45\textwidth}
		\includegraphics[width=\textwidth]{Funcional_Apertura_1.png}
		\caption{Apertura de mano para tomar objeto.}
		\label{Figura: Fun_A_1}
	\end{subfigure}
%	\hfill
	\begin{subfigure}[htbp]{0.45\textwidth}
		\includegraphics[width=\textwidth]{Funcional_Cierre.png}
		\caption{Pinza gruesa con objeto tomado.}
		\label{Figura: Fun_C}
	\end{subfigure}
%	\hfill
	\newline
	\begin{subfigure}[htbp]{0.45\textwidth}
		\includegraphics[width=\textwidth]{Funcional_Levantar.png}
		\caption{Levantamiento de objeto para trasladarlo.}
		\label{Figura: Fun_L}
	\end{subfigure}
%	\hfill
	\begin{subfigure}[htbp]{0.45\textwidth}
		\includegraphics[width=\textwidth]{Funcional_Apertura_2.png}
		\caption{Apertura de mano para soltar objeto posterior a su traslado.}
		\label{Figura: Fun_A_2}
	\end{subfigure}
	\caption{Fases de la tarea funcional asíncrona ejecutada por el sujeto (en línea).}
	\label{Figura: TareaFuncional}
\end{figure}
%Previo a realizar pruebas del esquema de control en línea, este se probó fuera de línea, aprovechando los registros de calibración. Para estas pruebas se diseñó un script en MATLAB que obtiene los parámetros necesarios del esquema de control de la misma forma que los arroja la calibración. Una vez obtenidos dichos parámetros se configura con ellos al esquema de control y se realiza una prueba fuera de línea donde con cada ventana de sEMG se obtiene un valor de RMS el cuál es sometido al esquema de control y arroja un valor de amplitud para el canal asociado al movimiento detectado. Tras probar el esquema de control con tres registros distintos de calibración se obtuvo un porcentaje de acierto del 81$\%$ en la identificación correcta de los movimientos de cierre, apertura y descanso de mano.

%En la Figura \ref{Figura: MapOff} se muestra el resultado de  una prueba exitosa del esquema de control fuera de línea, donde se observa que el esquema de control diseñado suele presentar errores en la identificación de los segmentos iniciales y finales de la tarea apertura de mano.



%\newpage
%Para la prueba en línea se configuró el modelo de Simulink con los datos obtenidos tras la calibración, y se solicitó al sujeto realizar el seguimiento de un par de señales trapezoidales que le indicarían el tipo de movimiento que tendría que lograr. Cuando la trapezoidal estuviera en cero, tendría que mantenerse en descanso; en la pendiente positiva de la trapezoidal tendría que realizar una transición de descanso hacia el movimiento completo solicitado; en la meseta de la trapezoidal tendría que mantener el movimiento completo solicitado; y en la pendiente negativa de la trapezoidal tendría que realizar una transición del movimiento completo solicitado hacia descanso.

%En la Figura \ref{Figura: MapOn} se muestra un segmento de una de las pruebas exitosas realizadas en línea. En dicha figura se puede observar que existe un retardo entre la trapezoidal y la respuesta del sistema de control, el cual es la suma del retardo que genera el procesamiento de la señal, el retardo ocasionado por el esquema de control, y el tiempo de respuesta del sujeto a la indicación de la trapezoidal.



%Para obtener el valor del retardo total se midió el tiempo existente entre el inicio de la pendiente positiva de la señal indicadora (trapezoidal) y la activación de la estimulación eléctrica. Al promediar los tiempos obtenidos a lo largo de las pruebas realizadas en línea se obtuvo un valor de 2.3 $\pm$ 0.3553 s.

%En la Figura \ref{Figura: Retardo} se muestra un acercamiento a las señales obtenidas en una prueba representativa de las pruebas realizadas en línea. Se muestran una sobre otra para visualizar el retardo existente entre el inicio de la señal indicadora y la activación de la estimulación eléctrica.





\chapter{Discusión}
%%DISCUSIÓN
El trabajo realizado en este proyecto se espera sea tomado en cuenta para generar mejores técnicas para rehabilitación de pacientes con discapacidad de miembro superior, basando dichas técnicas en una neuroprótesis para rehabilitación basada en FES. Este proyecto plantea las bases para realizar una adecuada implementación de una aplicación de estimulación eléctrica en lazo cerrado, a partir de señales sEMG del miembro no afectado en pacientes con hemiplejia por ACV (Accidente Cerebro-Vascular). Los principales resultados obtenidos se discuten a continuación, junto a las limitaciones del trabajo y posibles temas para trabajar a futuro.

\subsubsection*{Adquisición y decodificación de datos}
El subsistema diseñado para realizar la adquisición y decodificación de datos tiene la particularidad de que puede ser utilizado para cualquier dispositivo de adquisición que utilice un chip ADS1299, como el prototipo de adquisición en desarrollo en la División de Investigación Médica del INR-LGII, sólo son necesarios pequeños cambios en la selección de los bytes correspondientes a cada canal. Un problema que tiene este subsistema se encuentra en el bloque responsable de realizar la solicitud de muestras al dispositivo de adquisición, ya que es un bloque perteneciente al \emph{Instrument Control Toolbox} de Simulink\textregistered, por lo cual si no se cuenta con dicho toolbox el sistema no será funcional. Una posible mejora a este subsistema sería el diseño de un bloque responsable de la solicitud de muestras implementado en algún lenguaje de bajo nivel, esto podría hacer al sistema flexible y veloz, ya que actualmente el bloque de solicitud realiza una comunicación con MATLAB\textregistered para poder establecer una conexión serial con el dispositivo de adquisición, proceso que puede estar generando algún retraso dentro de todo el sistema. Un aspecto importante que podría ayudar a rastrear el origen del retardo existente actualmente en el sistema desarrollado en este proyecto, es la medición del retardo que genera dicho subsistema por sí solo, esto podría ayudar a determinar los puntos de trabajo para una mejora de este sistema desarrollado.

\subsubsection*{Protocolo para registro de sEMG}
El protocolo descrito en este proyecto se presta a errores humanos al momento de ubicar el lugar adecuado para la colocación de electrodos, por lo cual no se garantiza una repetibilidad del 100$\%$ en los registros. Se propone realizar un estudio donde se analice la actividad mioeléctrica en distintas posiciones del brazo en diversos sujetos, buscando obtener una estandarización en el posicionamiento de electrodos para aplicaciones similares a la desarrollada en este proyecto.

\subsubsection*{Procesamiento de sEMG}
Actualmente todo el procesamiento de las señales de sEMG se lleva a cabo por ventanas no traslapadas de adquisición, este proceso genera un retardo natural definido por la longitud de la ventana analizada, por lo cual el realizar un procesamiento con ventanas traslapadas o bien muestra a muestra podría disminuir este retardo natural. La implementación de un filtro de mediana móvil resultó de gran utilidad para conseguir una envolvente suave que sirviera como señal de control, sin embargo existen métodos como la regla trapezoidal que podrían arrojar resultados similares, por lo cual se podrían implementar algún otro método y determinar si dicho método disminuye el retardo del sistema. Un aspecto sumamente importante en el procesamiento es el hecho de que no se trabajó con señales de sEMG normalizadas, esto podría estar afectando al desempeño del sistema y sería una buena idea implementar una aplicación similar evaluando el desempeño utilizando sEMG normalizado y no normalizado. Esto último es particularmente importante, ya que la actividad mioeléctrica que presenta un sujeto en diversas sesiones puede cambiar de manera significativa, y claramente la actividad mioeléctrica tampoco será similiar entre diferentes sujetos, por lo que implementar el sistema utilizando señales de sEMG normalizado podría ayudar a estandarizar el sistema de control diseñado.

\subsubsection*{Sistema de control}
Recordando que el sistema de control consta de dos grandes partes: 1)Máquina de estados finitos para la identificación de movimientos; 2)Control proporcional para realizar modulación sEMG-FES. Se puede comentar lo siguiente:

%Actualmente el esquema de control se divide en dos grandes partes: 1)Máquina de estados finitos para la identificación de movimientos; 2)Ecuación lineal para realizar el mapeo sEMG-FES.

En cuanto a la identificación de movimientos, se considera que la implementación de un clasificador basado en una FSM que cambia de estado según se superen determinados valores de umbrales no es la mejor forma para realizar una clasificación, pero quizás sí una de las más fáciles. Este clasificador demostró ciertos problemas en identificar cambios de estado visualmente notorios, por ejemplo, en los movimientos incompletos de apertura o cierre existían momentos en los cuales el clasificador no identificaba el movimiento de forma adecuada a pesar de que visualmente se notara un cambio en la envolvente. Adicional a esto, existe también la posibilidad de que los propios umbrales puedan estar generando un retardo en el tiempo de inicio de la estimulación eléctrica, ya que habrá movimientos incompletos que no logren pasar ese umbral, y por lo tanto no activar la estimulación eléctrica; una buena idea sería probar disminuir los umbrales para lograr la activación de la estimulación con movimientos ligeros. Otro aspecto importante de este clasificador implementado, es la imposibilidad de compensar la fatiga muscular. Implementar un clasificador robusto basado en algún algoritmo de inteligencia artificial o LDA podría ser de mayor utilidad para una aplicación que se fuera utilizar para rehabilitación.

En cuanto al control proporcional encargado de la modulación de la amplitud FES, hay que destacar que este se obtiene a partir de dos umbrales obtenidos en la etapa de calibración, y considerando que el sEMG no presenta un comportamiento lineal, es probable que este método para la obtención de la ecuación del control proporcional no pueda realizar un seguimiento preciso a los cambios existentes en las señales de sEMG. Una mejora a esta parte podría ser la realización de una calibración con más de dos puntos, y a partir de ellos obtener una regresión lineal, o en su momento introducir algún bloque de control que logre representar la relación existente entre las señales sEMG y la fuerza ejercida o el torque muscular. Cabe destacar que debido a que la técnica de retroalimentación utilizada para este proyecto es el biofeedback, no existe como una señal de retroalimentación que se esté considerando dentro del control proporcional, y recordemos que la definición de este control estable que la señal control debe ser una señal de error obtenida a partir de la diferencia entre la señal ideal y la señal real, por ejemplo, la señal de sEMG del miembro no afectado y la señal de sEMG del miembro afectado. El agregar una señal de retroalimentación que pueda ser considerada dentro del sistema de control probablemente mejorará el desempeño de este.

Relacionado a la prueba de la tarea funcional, el hecho de que se lograra llevar a cabo con éxito es un indicador del potencial que tienen aplicaciones basadas en este proyecto para terapias de rehabilitación, dando a los pacientes la posibilidad de realizar tareas comunes de su vida diaria. Se espera que con las correctas adecuaciones, el sistema diseñado en este proyecto pueda ser de utilidad en sujetos con hemiplejia. Adicionalmente, sería bueno realizar una variante de esta prueba donde se llevara a cabo el mismo procedimiento pero mediante un experimento cruzado (un sujeto controla mientras otro sujeto es estimulado), y junto con algún algoritmo de visión por computadora poder determinar el retardo existente en esta prueba, para así tener una métrica más sobre el desempeño del sistema.

%Cabe destacar que al final del desarrollo de este proyecto se tuvo la posibilidad de probar la utilidad del sistema dentro de una situación de la vida real. Esta prueba se realizó a un sujeto sano, al cual se le otorgó la tarea de tomar un objeto cilíndrico con su mano derecha pero utilizando solamente la corriente eléctrica modulada por su brazo izquierdo. El sujeto pudo tomar el objeto de forma eficaz, y realizando movimientos del hombro logró levantar el objeto y trasladarlo a un lugar diferente a donde tomó el objeto. Con esto se logró demostrar que aplicaciones basadas en este proyecto pueden ser de utilidad en terapias de rehabilitación, dando a los pacientes la posibilidad de realizar tareas comunes de su vida diaria. Se espera que con las correctas adecuaciones, el sistema diseñado en este proyecto pueda ser de utilidad en sujetos con hemiplejia.

%Se ha demostrado que el sistema implementado en Simulink para realizar la adquisición y decodificación funciona adecuadamente, y gracias a la métrica de correlación obtenida se puede fundamentar su buen funcionamiento.
%
%El protocolo de registro, a pesar de no ser selectivo en su totalidad, permite distinguir en la mayoría de los casos entre actividad sEMG relacionada a un movimiento u otro. Sin embargo, se ha observado que se deben de dar indicaciones claras al sujeto experimental, ya que se a observado que al realizar contracciones fuertes en cualquier movimiento, la actividad de sEMG en ambos canales es muy similar.
%
%El esquema de filtrado diseñado para los registros de sEMG es útil ya que permite obtener una señal de sEMG en la que se pueden distinguir las contracciones de los momentos de reposo. En cuanto a los descriptores de amplitud se optará por utilizar el valor RMS debido a que proporciona valores más grandes que el valor ARV/MAV. Cabe destacar que se ha observado que, en cuanto a morfología de las señales obtenidas tras aplicar el descriptor, se obtiene una forma muy similar para ambos descriptores. Un punto importante relacionado al procesamiento es el método de suavizado; este genera una onda suave que permite obtener un mapeo con mejor estabilidad que sin suavizar, sin embargo, este procedimiento de suavizado utiliza un buffer de 10 muestras de valor RMS, recordemos que cada muestra de RMS se obtiene a partir del procesamiento de ventanas de 0.1 segundos de sEMG, lo cuál implica que se está generando un retraso total de 1 segundo en la actualización del parámetro a mapear, lo cual nos coloca en el límite del tiempo prometido; para solucionar esto se probaran ventanas más cortas.
%
%El mapeo lineal es útil hasta cierto punto, pero si se opta por utilizar dicho mapeo es muy probable que sea necesario utilizar una lógica de control adicional que ayude a obtener un mapeo más estable. Adicional a esto aún está como alternativa a probar el obtener una relación entre los descriptores y la amplitud de la corriente de estimulación a través de una calibración a 5 posturas de la mano (cierre total de la mano, cierre parcial, descanso, apertura parcial de mano y apertura total).
%
%Hay que destacar que todo el trabajo realizado se ha probado de forma exitosa con el dispositivo OpenBCI debido a que el prototipo presenta problemas en el circuito encargado de la conversión analógico-digital. Afortunadamente todo lo desarrollado tiene la posibilidad de migrarse al dispositivo de forma sencilla, por lo cual en cuanto se solucionen los problemas con el prototipo se migrará todo a este.

\renewcommand{\bibname}{Referencias}
\bibliographystyle{acm}
\bibliography{11_biblio}

\end{document}