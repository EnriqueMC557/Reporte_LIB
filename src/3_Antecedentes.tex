\section{Desarrollos previos al proyecto}
En el INR-LGII se han realizado trabajos previos relacionados al desarrollo de una neuroprótesis, los cuales han logrado que dicho sistema sea funcional, e incluso se ha probado con algunos pacientes del propio instituto. Estos trabajos incluyen una plataforma de software para control y configuración de la neuroprótesis \cite{Fuentes2018}\cite{JanethFuentes2018} y la implementación de una aplicación FES en lazo abierto comandada por EEG\cite{Castillo2019}\cite{OmarCastillo2019}.

\subsection{Plataforma de software para neuroprótesis}
Consiste en una GUI implementada en MATLAB\textregistered \; (The MathWorks Inc., Natick, MA, E.U.A.), la cual consta de 4 pantallas que en conjunto permiten: a) realizar el registro de datos de un paciente o usuario en el que se probará el dispositivo, b) realizar el entrenamiento de un clasificador de movimientos voluntarios, c) ejecutar una aplicación FES en lazo abierto, o bien d) experimentar con los parámetros del estimulador y el sistema de registro de biopotenciales para determinar el patrón de estimulación óptimo para el paciente. Esta plataforma realiza una conexión a dispositivos comerciales: RehaStim 2 para estimulación eléctrica, y Cyton Board para adquisición de biopotenciales) que permiten la integración de las funciones de la neuroprotesis \cite{Fuentes2018}\cite{JanethFuentes2018}.

\subsection{Aplicación FES en lazo abierto}
La aplicación FES, que se encuentra inmersa en la plataforma de software para la neuroprótesis, está basada en una Interfaz Cerebro-Computadora. Dicha aplicación le muestra al sujeto una serie de 5 movimientos predefinidos, dentro de los cuales el sujeto debe seleccionar alguno cerrando los ojos. Una vez seleccionado y confirmado el movimiento objetivo, el sistema envía una secuencia de pulsos de estimulación eléctrica para asistir al sujeto a realizar el movimiento elegido. En esta aplicación los parámetros de estimulación eléctrica están predeterminados antes de iniciar la aplicación, los cuales se determinan en una sesión de calibración personalizada \cite{Castillo2019}\cite{OmarCastillo2019}.


\section{Sistemas FES existentes}
En la literatura existe una diversidad de trabajos que implementan un lazo cerrado para aplicaciones FES, los cuales utilizan dispositivos de estimulación que varían entre dispositivos comerciales o prototipos, sin embargo, la revisión bibliográfica realizada para este proyecto se centró en trabajos que utilizan el dispositivo RehaStim como dispositivo de estimulación, y que además implementaran alguna aplicación para rehabilitación de miembro superior.

Trabajos como los documentados en \cite{Salchow2016} y \cite{Kim2015}, muestran la importancia de las aplicaciones que implementan una terapia por medio de un entrenamiento en espejo para facilitar la recuperación motora de miembros superiores e inferiores en pacientes con hemiplejia. Dicha importancia radica en lograr la ilusión de un movimiento sincrónico entre dos extremidades sanas, ilusión que ha demostrado puede promover la recuperación de la funcionalidad de la extremidad paralizada \cite{Deconinck2015}.

El trabajo documentado en \cite{Sun2014} demuestra la gran capacidad que tienen los algoritmos implementados en una máquina de estados finitos para realizar el control de una neuroprótesis, además de demostrar que estos algoritmos permiten al experimentador una comprensión rápida sobre el funcionamiento del esquema de control.

Otros trabajos como lo son \cite{Simonsen2017} y \cite{Woods2018} son de utilidad para el proyecto debido a que demuestran que al lograr una integración de los componentes y control de una neuroprótesis se pueden realizar aplicaciones que presenten un funcionamiento en tiempo real o muy cercano a este.

El Cuadro \ref{Cuadro:Sistemas FES} resume la información de los trabajos mencionados anteriormente, mostrando información como autor, año de publicación, propósito del trabajo, señales de comando y retroalimentación utilizadas, y dispositivo de estimulación. Cabe mencionar que todos estos trabajos presentan una implementación dentro de las plataformas de software MATLAB\textregistered \; y Simulink\textregistered.

%Cuadro de revisión bibliográfica
\begin{table}[hbt]
	\centering
	\begin{tabular}{|p{30mm}|p{10mm}|p{40mm}|p{40mm}|p{25mm}|}
	\hline
	\textbf{Autor} & \textbf{Año} & \textbf{Propósito} & \textbf{Señales utilizadas} & \textbf{Dispositivo}\\ 
	\hline
	\hline
	Christina Salchow, \emph{et al.} \cite{Salchow2016} & 2016 & Entrenamiento en espejo aplicado a la mano & Electromiografía y movimiento de mano & RehaMove Pro\\
	\hline
	Mignxu Sun \cite{Sun2014} & 2014 & Recuperación de funciones de miembro superior & Acelerometría & RehaStim 1\\
	\hline
	Daniel Simonsen, \emph{et al.} \cite{Simonsen2017} & 2017 & Asistencia para apertura y cierre de mano & Posición del objeto y posición de la mano & STMISOLA\\
	\hline
	Billy Woods, \emph{et al.} \cite{Woods2018} & 2018 & Asistencia en miembro inferior para funciones de ciclismo & Mecanomiografía, fuerza aplicada a pedales y posición de cigüeñal & RehaStim 1\\
	\hline
	\end{tabular}
	\caption{Revisión de sistemas FES reportados en la literatura con aplicaciones similares a las de este proyecto.}
	\label{Cuadro:Sistemas FES}
\end{table}

Una revisión bibliográfica adicional se centró en trabajos que utilizaran señales de sEMG como señal de control, para rescatar los descriptores de amplitud comúnmente usados y los tipos de control utilizados para cada aplicación.

El Cuadro \ref{Cuadro:Control} resumen la información de los trabajos consultados para dicha revisión bibliográfica, mostrando información como autor, año de publicación, propósito de la aplicación, descriptor de amplitud utilizado y tipo de control implementado.

\begin{table}[htb]
	\centering
	\begin{tabular}{|p{30mm}|p{10mm}|p{45mm}|p{25mm}|p{35mm}|}
	\hline
	\textbf{Autor} & \textbf{Año} & \textbf{Propósito} & \textbf{Descriptor sEMG} & \textbf{Tipo de control}\\
	\hline
	\hline
	Yu Zhou, \emph{et al.} \cite{Zhou2018} & 2018 & FES contralateral para miembro superior & RMS & Regresión lineal\\
	\hline
	Tommaso Lenzi, \emph{et al.} \cite{Lenzi2012} & 2012 & Control de exoesqueleto & Envolvente lineal & Proporcional\\
	\hline
	Jung Hee Kim, \emph{et al.} \cite{Kim2015} & 2015 & Terapia en espejo para recuperación de miembro superior & RMS & On-Off\\
	\hline
	Gustavo Aguirre, \emph{et al.} \cite{Aguirre-Vargas2015} & 2015 & Control de brazo robótico & Amplitud & Lógica difusa\\
	\hline
	Xin Yi, \emph{et al.} \cite{Yi2013} & 2013 & Cierre contralateral de párpado en conejos & RMS & On-Off\\
	\hline
	Sachs NA, \emph{et al.} \cite{Sachs2006} & 2006 & Cierre contralateral de párpado en roedores & Integración & On/Off\\
	\hline
	Lucas Fonseca, \emph{et al.} \cite{Fonseca2019} & 2019 & Asistencia contralateral para cierre de mano & Envolvente & FSM\\
	\hline
	\end{tabular}
	\caption{Revisión de tipos de control y descriptores de amplitud comúnmente usados en aplicaciones de control basado en sEMG.}
	\label{Cuadro:Control}
\end{table}
