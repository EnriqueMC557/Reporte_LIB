%{\color{red}Peña: Falta respaldo de afirmaciones con investigación documental}

Con el desarrollo de este proyecto se podrá lograr una intervención basada en la actividad, que permita la repetición de movimientos en un contexto relevante para la rehabilitación del sujeto, y donde se involucre la actividad voluntaria del sujeto, y esta a su vez tenga un efecto en la estimulación eléctrica aplicada y el movimiento resultante. De este modo la operación del sistema ya no se llevará a cabo con una modulación subjetiva de los parámetros de estimulación, dotando a los pacientes del control de los movimientos de su rehabilitación.

Esta propuesta de enfoque en lazo cerrado también podrá contribuir a la seguridad para el usuario; ya que, hasta ahora, debido a la modificación experimentador-dependiente de parámetros por parte del experimentador, algunos usuarios han reportado molestias ante incrementos en la intensidad de la estimulación entre repeticiones de los movimientos generados por FES.

Por último, este proyecto busca reducir la dependencia del experimentador al momento de realizar la estimulación, a través del desarrollo de aplicaciones en lazo cerrado retroalimentados, que midan de manera objetiva el resultado de la estimulación, y mediante algoritmos de procesamiento y modelos de control en línea, contribuyan a la participación del usuario en su propia rehabilitación.